\documentclass[]{article}
\usepackage{lmodern}
\usepackage{amssymb,amsmath}
\usepackage{ifxetex,ifluatex}
\usepackage{fixltx2e} % provides \textsubscript
\ifnum 0\ifxetex 1\fi\ifluatex 1\fi=0 % if pdftex
  \usepackage[T1]{fontenc}
  \usepackage[utf8]{inputenc}
\else % if luatex or xelatex
  \ifxetex
    \usepackage{mathspec}
  \else
    \usepackage{fontspec}
  \fi
  \defaultfontfeatures{Ligatures=TeX,Scale=MatchLowercase}
\fi
% use upquote if available, for straight quotes in verbatim environments
\IfFileExists{upquote.sty}{\usepackage{upquote}}{}
% use microtype if available
\IfFileExists{microtype.sty}{%
\usepackage{microtype}
\UseMicrotypeSet[protrusion]{basicmath} % disable protrusion for tt fonts
}{}
\usepackage[margin=1in]{geometry}
\usepackage{hyperref}
\hypersetup{unicode=true,
            pdftitle={Isotope Tracking - Species based},
            pdfborder={0 0 0},
            breaklinks=true}
\urlstyle{same}  % don't use monospace font for urls
\usepackage{natbib}
\bibliographystyle{plainnat}
\usepackage{color}
\usepackage{fancyvrb}
\newcommand{\VerbBar}{|}
\newcommand{\VERB}{\Verb[commandchars=\\\{\}]}
\DefineVerbatimEnvironment{Highlighting}{Verbatim}{commandchars=\\\{\}}
% Add ',fontsize=\small' for more characters per line
\usepackage{framed}
\definecolor{shadecolor}{RGB}{248,248,248}
\newenvironment{Shaded}{\begin{snugshade}}{\end{snugshade}}
\newcommand{\KeywordTok}[1]{\textcolor[rgb]{0.13,0.29,0.53}{\textbf{{#1}}}}
\newcommand{\DataTypeTok}[1]{\textcolor[rgb]{0.13,0.29,0.53}{{#1}}}
\newcommand{\DecValTok}[1]{\textcolor[rgb]{0.00,0.00,0.81}{{#1}}}
\newcommand{\BaseNTok}[1]{\textcolor[rgb]{0.00,0.00,0.81}{{#1}}}
\newcommand{\FloatTok}[1]{\textcolor[rgb]{0.00,0.00,0.81}{{#1}}}
\newcommand{\ConstantTok}[1]{\textcolor[rgb]{0.00,0.00,0.00}{{#1}}}
\newcommand{\CharTok}[1]{\textcolor[rgb]{0.31,0.60,0.02}{{#1}}}
\newcommand{\SpecialCharTok}[1]{\textcolor[rgb]{0.00,0.00,0.00}{{#1}}}
\newcommand{\StringTok}[1]{\textcolor[rgb]{0.31,0.60,0.02}{{#1}}}
\newcommand{\VerbatimStringTok}[1]{\textcolor[rgb]{0.31,0.60,0.02}{{#1}}}
\newcommand{\SpecialStringTok}[1]{\textcolor[rgb]{0.31,0.60,0.02}{{#1}}}
\newcommand{\ImportTok}[1]{{#1}}
\newcommand{\CommentTok}[1]{\textcolor[rgb]{0.56,0.35,0.01}{\textit{{#1}}}}
\newcommand{\DocumentationTok}[1]{\textcolor[rgb]{0.56,0.35,0.01}{\textbf{\textit{{#1}}}}}
\newcommand{\AnnotationTok}[1]{\textcolor[rgb]{0.56,0.35,0.01}{\textbf{\textit{{#1}}}}}
\newcommand{\CommentVarTok}[1]{\textcolor[rgb]{0.56,0.35,0.01}{\textbf{\textit{{#1}}}}}
\newcommand{\OtherTok}[1]{\textcolor[rgb]{0.56,0.35,0.01}{{#1}}}
\newcommand{\FunctionTok}[1]{\textcolor[rgb]{0.00,0.00,0.00}{{#1}}}
\newcommand{\VariableTok}[1]{\textcolor[rgb]{0.00,0.00,0.00}{{#1}}}
\newcommand{\ControlFlowTok}[1]{\textcolor[rgb]{0.13,0.29,0.53}{\textbf{{#1}}}}
\newcommand{\OperatorTok}[1]{\textcolor[rgb]{0.81,0.36,0.00}{\textbf{{#1}}}}
\newcommand{\BuiltInTok}[1]{{#1}}
\newcommand{\ExtensionTok}[1]{{#1}}
\newcommand{\PreprocessorTok}[1]{\textcolor[rgb]{0.56,0.35,0.01}{\textit{{#1}}}}
\newcommand{\AttributeTok}[1]{\textcolor[rgb]{0.77,0.63,0.00}{{#1}}}
\newcommand{\RegionMarkerTok}[1]{{#1}}
\newcommand{\InformationTok}[1]{\textcolor[rgb]{0.56,0.35,0.01}{\textbf{\textit{{#1}}}}}
\newcommand{\WarningTok}[1]{\textcolor[rgb]{0.56,0.35,0.01}{\textbf{\textit{{#1}}}}}
\newcommand{\AlertTok}[1]{\textcolor[rgb]{0.94,0.16,0.16}{{#1}}}
\newcommand{\ErrorTok}[1]{\textcolor[rgb]{0.64,0.00,0.00}{\textbf{{#1}}}}
\newcommand{\NormalTok}[1]{{#1}}
\usepackage{graphicx,grffile}
\makeatletter
\def\maxwidth{\ifdim\Gin@nat@width>\linewidth\linewidth\else\Gin@nat@width\fi}
\def\maxheight{\ifdim\Gin@nat@height>\textheight\textheight\else\Gin@nat@height\fi}
\makeatother
% Scale images if necessary, so that they will not overflow the page
% margins by default, and it is still possible to overwrite the defaults
% using explicit options in \includegraphics[width, height, ...]{}
\setkeys{Gin}{width=\maxwidth,height=\maxheight,keepaspectratio}
\IfFileExists{parskip.sty}{%
\usepackage{parskip}
}{% else
\setlength{\parindent}{0pt}
\setlength{\parskip}{6pt plus 2pt minus 1pt}
}
\setlength{\emergencystretch}{3em}  % prevent overfull lines
\providecommand{\tightlist}{%
  \setlength{\itemsep}{0pt}\setlength{\parskip}{0pt}}
\setcounter{secnumdepth}{0}
% Redefines (sub)paragraphs to behave more like sections
\ifx\paragraph\undefined\else
\let\oldparagraph\paragraph
\renewcommand{\paragraph}[1]{\oldparagraph{#1}\mbox{}}
\fi
\ifx\subparagraph\undefined\else
\let\oldsubparagraph\subparagraph
\renewcommand{\subparagraph}[1]{\oldsubparagraph{#1}\mbox{}}
\fi

%%% Use protect on footnotes to avoid problems with footnotes in titles
\let\rmarkdownfootnote\footnote%
\def\footnote{\protect\rmarkdownfootnote}

%%% Change title format to be more compact
\usepackage{titling}

% Create subtitle command for use in maketitle
\newcommand{\subtitle}[1]{
  \posttitle{
    \begin{center}\large#1\end{center}
    }
}

\setlength{\droptitle}{-2em}
  \title{Isotope Tracking - Species based}
  \pretitle{\vspace{\droptitle}\centering\huge}
  \posttitle{\par}
  \author{}
  \preauthor{}\postauthor{}
  \date{}
  \predate{}\postdate{}

\usepackage{mathtools} \usepackage{natbib}

\begin{document}
\maketitle

To obtain the masses of heavy \(M^h\) and light isotopes \(M^l\), the
isotope signature \(\delta ^{13}C\) is used where:

\[
\delta ^{13}C [‰] = \Big(\frac{R_{smp} - R_{std}}{R_{std}}\Big)\cdot 1000
\]

\[
R_{smp} = \frac{M^h}{M^l} 
\]

\[
R_{std} = 11237.2 \cdot 10^{-6}
\]

and where the total mass \(M_{tot} = M^h + M^l\), we obtain:

\[
R_{std}\Big(\delta ^{13}C/1000+1 \Big)= \frac{M_{tot}-M^l}{M^l}
\]

\[
M^l = \frac{M_{tot}}{1+R_{std}(\delta ^{13}C/1000+1)}
\]

\[
M^h = M_{tot} - M^l 
\]

\subsubsection{Volatilization}\label{volatilization}

To calculate the fraction volatilized into the atmosphere the partial
pressure \(p\) \((atm)\) of S-met in pore space is obtained from the
aqueous concentration \(C_{aq}\) \((\mu g~L^{-1})\) \textit{via} the
physical Henry law constant \(H^{cp} = 42.6 \pm 2.8 \cdot10^3\)
\((mol~L^{-1} atm^{-1})\) \citep{Feigenbrugel2004}, where:

\[
p = \frac{C_{aq}}{H^{cp}}
\] The partial pressure is then converted to moles \textit{via} the
ideal gas law:

\[
pV = nRT
\] where the volume \(V\) \((L)\) considered is that of the total
porespace of the first 10 mm top soil, the gas constant
\(R = 0.0821~atm~L~mol^{-1}K^{-1}\), and soil temperature \(T\)
(Kelvin). The concentrtion in gas pore space is obtained then as:

\begin{Shaded}
\begin{Highlighting}[]

\KeywordTok{def} \NormalTok{getGasHenry(model, conc_aq, theta_gas):}
    \NormalTok{R }\OperatorTok{=} \NormalTok{scalar(}\FloatTok{0.0821}\NormalTok{)  }\CommentTok{# atm L/mol Kelvin}
    \NormalTok{T }\OperatorTok{=} \NormalTok{model.temp_z0_fin }\OperatorTok{+} \FloatTok{273.15}  \CommentTok{# Kelvin}
    \NormalTok{Vair }\OperatorTok{=} \NormalTok{model.z0}\OperatorTok{*}\NormalTok{cellarea()}\OperatorTok{*}\NormalTok{theta_gas  }\CommentTok{# mm*m2 = L}
    \CommentTok{# Convert to mol/L}
    \CommentTok{# mol/L = ug/L * (1g/10**6 ug)* 1mol/283.7g}
    \NormalTok{conc_M }\OperatorTok{=} \NormalTok{conc_aq }\OperatorTok{*} \DecValTok{1}\OperatorTok{/}\DecValTok{10}\OperatorTok{**}\DecValTok{6} \OperatorTok{*} \DecValTok{1}\OperatorTok{/}\NormalTok{model.molar  }\CommentTok{# mol/L}
    \NormalTok{p }\OperatorTok{=} \NormalTok{conc_M }\OperatorTok{/} \NormalTok{model.k_cp  }\CommentTok{# atm}
    \NormalTok{n }\OperatorTok{=} \NormalTok{p}\OperatorTok{*}\NormalTok{Vair}\OperatorTok{/}\NormalTok{(R}\OperatorTok{*}\NormalTok{T)  }\CommentTok{# moles}
    \NormalTok{conc_gas }\OperatorTok{=} \NormalTok{n}\OperatorTok{*}\NormalTok{model.molar}\OperatorTok{*}\DecValTok{10}\OperatorTok{**}\DecValTok{6}\OperatorTok{/}\NormalTok{Vair}
    \ControlFlowTok{return} \NormalTok{conc_gas  }\CommentTok{# ug/L}
    
\end{Highlighting}
\end{Shaded}

Based on the porespace concentration, the mass flux escaping into the
atmosphere is then obtained as:

\begin{Shaded}
\begin{Highlighting}[]

\KeywordTok{def} \NormalTok{getVolatileMass(model, temp_air, theta_sat,}
                    \NormalTok{rel_diff_model}\OperatorTok{=}\StringTok{"option-1"}\NormalTok{, sorption_model}\OperatorTok{=}\StringTok{"linear"}\NormalTok{,}
                    \NormalTok{gas}\OperatorTok{=}\VariableTok{True}\NormalTok{, isotopes}\OperatorTok{=}\VariableTok{True}\NormalTok{):}
    \CommentTok{# Volatilize only during peak volatilization time i.e., first 24 hrs, @Prueger2005.}
    \NormalTok{theta_layer }\OperatorTok{=} \NormalTok{model.theta_z0}
    \NormalTok{theta_gas }\OperatorTok{=} \BuiltInTok{max}\NormalTok{(theta_sat }\OperatorTok{-} \NormalTok{theta_layer, scalar(}\DecValTok{0}\NormalTok{))}
    \NormalTok{tot_mass }\OperatorTok{=} \NormalTok{model.pestmass_z0  }\CommentTok{# ug}
    \NormalTok{light_mass }\OperatorTok{=} \NormalTok{model.lightmass_z0  }\CommentTok{# ug}
    \NormalTok{heavy_mass }\OperatorTok{=} \NormalTok{model.heavymass_z0  }\CommentTok{# ug}
    \CommentTok{# Convert to m (needed for final mass computation on cell basis)}
    \NormalTok{depth_m }\OperatorTok{=} \NormalTok{model.z0 }\OperatorTok{*} \DecValTok{1} \OperatorTok{/} \DecValTok{10} \OperatorTok{**} \DecValTok{3}
    \CommentTok{# Air boundary layer, assumed as 2m high}
    \NormalTok{thickness_a }\OperatorTok{=} \NormalTok{scalar(}\FloatTok{2.0}\NormalTok{)  }\CommentTok{# m}
    \CommentTok{# Diffusion coefficient in air (cm^2/s); https://www.gsi-net.com}
    \CommentTok{#  D_ar (metolachlor) = 0.03609052694,  at reference Temp., in Kelvin, D_a,r)}
    \NormalTok{diff_ar }\OperatorTok{=} \FloatTok{0.03609} \OperatorTok{*} \DecValTok{86400} \OperatorTok{*} \DecValTok{1} \OperatorTok{/} \DecValTok{10} \OperatorTok{**} \DecValTok{4}  \CommentTok{# m2/d}
    \CommentTok{# Diffusion coefficient adjusted to air Temp. in Kelvin, D_a}
    \NormalTok{diff_a }\OperatorTok{=} \NormalTok{((temp_air }\OperatorTok{+} \FloatTok{273.15}\NormalTok{) }\OperatorTok{/} \FloatTok{293.15}\NormalTok{) }\OperatorTok{**} \FloatTok{1.75} \OperatorTok{*} \NormalTok{diff_ar  }\CommentTok{# m2/d}

    \ControlFlowTok{if} \NormalTok{rel_diff_model }\OperatorTok{==} \StringTok{"option-1"}\NormalTok{:}
        \CommentTok{# Millington and Quirk, 1960 (in Leistra, 2001, p.48)}
        \CommentTok{# a,b parameters: Jin and Jury, 1996 (in Leistra, 2001)}
        \NormalTok{diff_relative_gas }\OperatorTok{=} \NormalTok{(diff_a }\OperatorTok{*} \NormalTok{theta_gas }\OperatorTok{**} \DecValTok{2} \OperatorTok{/}
                             \NormalTok{theta_sat }\OperatorTok{**} \NormalTok{(}\DecValTok{2} \OperatorTok{/} \DecValTok{3}\NormalTok{))  }\CommentTok{# m2/d}
    \ControlFlowTok{elif} \NormalTok{rel_diff_model }\OperatorTok{==} \StringTok{"option-2"}\NormalTok{:}
        \CommentTok{# Currie 1960 (in Leistra, 2001)}
        \CommentTok{# a,b parameters: Baker, 1987 (in Leistra, 2001)}
        \NormalTok{diff_relative_gas }\OperatorTok{=} \NormalTok{diff_a }\OperatorTok{*} \FloatTok{2.5} \OperatorTok{*} \NormalTok{theta_gas }\OperatorTok{**} \DecValTok{3}  \CommentTok{# m2/d}
    \ControlFlowTok{else}\NormalTok{:}
        \BuiltInTok{print}\NormalTok{(}\StringTok{"No appropriate relative diffusion parameter chosen"}\NormalTok{)}
        \NormalTok{diff_relative_gas }\OperatorTok{=} \NormalTok{diff_a  }\CommentTok{# m2/d}
    \CommentTok{# Transport resistance through air (r_a) and soil (r_s) layer}
    \NormalTok{r_a }\OperatorTok{=} \NormalTok{thickness_a }\OperatorTok{/} \NormalTok{diff_a  }\CommentTok{# d/m}
    \NormalTok{r_s }\OperatorTok{=} \NormalTok{(}\FloatTok{0.5} \OperatorTok{*} \NormalTok{depth_m) }\OperatorTok{/} \NormalTok{diff_relative_gas  }\CommentTok{# d/m}

    \CommentTok{# Retardation factor}
    \ControlFlowTok{if} \NormalTok{sorption_model }\OperatorTok{==} \StringTok{"linear"}\NormalTok{:}
        \NormalTok{retard_layer }\OperatorTok{=} \DecValTok{1} \OperatorTok{+} \NormalTok{(model.p_b }\OperatorTok{*} \NormalTok{model.k_d) }\OperatorTok{/} \NormalTok{theta_layer}
    \ControlFlowTok{else}\NormalTok{:}
        \NormalTok{retard_layer }\OperatorTok{=} \DecValTok{1}
    \CommentTok{# Aqueous concentration}
    \ControlFlowTok{if} \NormalTok{gas:}
        \CommentTok{# Leistra et al., 2001}
        \ControlFlowTok{if} \NormalTok{isotopes:}
            \NormalTok{conc_light_aq }\OperatorTok{=} \NormalTok{light_mass }\OperatorTok{/} \NormalTok{((cellarea() }\OperatorTok{*} \NormalTok{model.z0) }\OperatorTok{*}  \CommentTok{# m2 * mm = L}
                                          \NormalTok{(theta_gas }\OperatorTok{*} \NormalTok{model.k_h }\OperatorTok{+}
                                           \NormalTok{theta_layer }\OperatorTok{*} \NormalTok{retard_layer))  }\CommentTok{# ug/L}
            \NormalTok{conc_heavy_aq }\OperatorTok{=} \NormalTok{heavy_mass }\OperatorTok{/} \NormalTok{((cellarea() }\OperatorTok{*} \NormalTok{model.z0) }\OperatorTok{*}
                                          \NormalTok{(theta_gas }\OperatorTok{*} \NormalTok{model.k_h }\OperatorTok{+}
                                           \NormalTok{theta_layer }\OperatorTok{*} \NormalTok{retard_layer))  }\CommentTok{# ug/L}
        \ControlFlowTok{else}\NormalTok{:}
            \NormalTok{conc_layer_aq }\OperatorTok{=} \NormalTok{tot_mass }\OperatorTok{/} \NormalTok{((cellarea() }\OperatorTok{*} \NormalTok{model.z0) }\OperatorTok{*}
                                        \NormalTok{(theta_gas }\OperatorTok{*} \NormalTok{model.k_h }\OperatorTok{+}
                                         \NormalTok{theta_layer }\OperatorTok{*} \NormalTok{retard_layer))  }\CommentTok{# ug/L}
    \ControlFlowTok{else}\NormalTok{:  }\CommentTok{# No gas phase}
        \ControlFlowTok{if} \NormalTok{isotopes:}
            \NormalTok{conc_light_aq }\OperatorTok{=} \NormalTok{light_mass }\OperatorTok{/} \NormalTok{((cellarea() }\OperatorTok{*} \NormalTok{model.z0) }\OperatorTok{*}
                                          \NormalTok{(theta_layer }\OperatorTok{*} \NormalTok{retard_layer))  }\CommentTok{# ug/L}
            \NormalTok{conc_heavy_aq }\OperatorTok{=} \NormalTok{heavy_mass }\OperatorTok{/} \NormalTok{((cellarea() }\OperatorTok{*} \NormalTok{model.z0) }\OperatorTok{*}
                                          \NormalTok{(theta_layer }\OperatorTok{*} \NormalTok{retard_layer))  }\CommentTok{# ug/L}
        \ControlFlowTok{else}\NormalTok{:  }\CommentTok{# No gas phase, no isotopes considered}
            \CommentTok{# Whelan, 1987}
            \NormalTok{conc_layer_aq }\OperatorTok{=} \NormalTok{((tot_mass }\OperatorTok{/} \NormalTok{cellarea()) }\OperatorTok{/}
                             \NormalTok{(theta_layer }\OperatorTok{*} \NormalTok{retard_layer }\OperatorTok{*} \NormalTok{model.z0))  }\CommentTok{# ug/L}

    \CommentTok{# Convert ug/L to ug/m3, as will be multiplying by cell's area in m2}
    \ControlFlowTok{if} \NormalTok{isotopes:}
        \NormalTok{conc_gas_light }\OperatorTok{=} \NormalTok{getGasHenry(model, conc_light_aq, theta_gas) }\OperatorTok{*} \DecValTok{10}\OperatorTok{**}\DecValTok{3}  \CommentTok{# ug/L * 10^3 L/m3}
        \NormalTok{conc_gas_heavy }\OperatorTok{=} \NormalTok{getGasHenry(model, conc_heavy_aq, theta_gas) }\OperatorTok{*} \DecValTok{10}\OperatorTok{**}\DecValTok{3}
        \NormalTok{volat_flux_light }\OperatorTok{=} \NormalTok{(conc_gas_light }\OperatorTok{/} \NormalTok{(r_a }\OperatorTok{+} \NormalTok{r_s)) }\OperatorTok{*} \NormalTok{cellarea()  }\CommentTok{# ug/day}
        \NormalTok{volat_flux_heavy }\OperatorTok{=} \NormalTok{(conc_gas_heavy }\OperatorTok{/} \NormalTok{(r_a }\OperatorTok{+} \NormalTok{r_s)) }\OperatorTok{*} \NormalTok{cellarea()  }\CommentTok{# ug/day}
        \NormalTok{volat_flux }\OperatorTok{=} \NormalTok{volat_flux_light }\OperatorTok{+} \NormalTok{volat_flux_heavy}
    \ControlFlowTok{else}\NormalTok{:}
        \NormalTok{conc_gas }\OperatorTok{=} \NormalTok{getGasHenry(model, conc_layer_aq, theta_gas) }\OperatorTok{*} \DecValTok{10}\OperatorTok{**}\DecValTok{3}  \CommentTok{# ug/L * 10^3 L/m3}
        \NormalTok{volat_flux }\OperatorTok{=} \NormalTok{(conc_gas }\OperatorTok{/} \NormalTok{(r_a }\OperatorTok{+} \NormalTok{r_s)) }\OperatorTok{*} \NormalTok{cellarea()  }\CommentTok{# ug/day}
        \NormalTok{volat_flux_light }\OperatorTok{=} \VariableTok{None}
        \NormalTok{volat_flux_heavy }\OperatorTok{=} \VariableTok{None}

    \ControlFlowTok{return} \NormalTok{\{}\StringTok{"mass_volat"}\NormalTok{: volat_flux,}
            \CommentTok{"light_volat"}\NormalTok{: volat_flux_light,}
            \CommentTok{"heavy_volat"}\NormalTok{: volat_flux_heavy\}  }\CommentTok{# ug/dt}
            
\end{Highlighting}
\end{Shaded}

\section{Runoff Mass}\label{runoff-mass}

\begin{Shaded}
\begin{Highlighting}[]


\KeywordTok{def} \NormalTok{getRunOffMass(model, theta_sat, precip, runoff_mm,}
                  \NormalTok{transfer_model}\OperatorTok{=}\StringTok{"simple-mt"}\NormalTok{, sorption_model}\OperatorTok{=}\StringTok{"linear"}\NormalTok{,}
                  \NormalTok{gas}\OperatorTok{=}\VariableTok{True}\NormalTok{):}
    \NormalTok{depth, theta_layer, delta_layer }\OperatorTok{=} \NormalTok{model.z0, model.theta_z0, model.delta_z0}

    \ControlFlowTok{if} \NormalTok{sorption_model }\OperatorTok{==} \StringTok{"linear"}\NormalTok{:}
        \CommentTok{# Retardation factor}
        \NormalTok{retard_layer }\OperatorTok{=} \DecValTok{1} \OperatorTok{+} \NormalTok{(model.p_b }\OperatorTok{*} \NormalTok{model.k_d) }\OperatorTok{/} \NormalTok{theta_layer}
    \ControlFlowTok{else}\NormalTok{:}
        \NormalTok{retard_layer }\OperatorTok{=} \DecValTok{1}

    \CommentTok{# Aqueous concentration calculation}
    \ControlFlowTok{if} \NormalTok{gas:}
        \CommentTok{# Leistra et al., 2001}
        \NormalTok{theta_gas }\OperatorTok{=} \NormalTok{theta_sat }\OperatorTok{-} \NormalTok{model.theta_z0}
        \CommentTok{# TODO: Check that all theta_gas >= 0}
        \NormalTok{conc_layer_aq }\OperatorTok{=} \NormalTok{model.pestmass_z0 }\OperatorTok{/} \NormalTok{((cellarea() }\OperatorTok{*} \NormalTok{depth) }\OperatorTok{*}
                                             \NormalTok{(theta_gas }\OperatorTok{*} \NormalTok{model.k_h }\OperatorTok{+} \NormalTok{theta_layer }\OperatorTok{*} \NormalTok{retard_layer))  }\CommentTok{# mg/L}
    \ControlFlowTok{else}\NormalTok{:}
        \CommentTok{# Whelan, 1987 # No gas phase considered}
        \NormalTok{conc_layer_aq }\OperatorTok{=} \NormalTok{(model.pestmass_z0 }\OperatorTok{/} \NormalTok{cellarea()) }\OperatorTok{/} \NormalTok{(theta_layer }\OperatorTok{*} \NormalTok{retard_layer }\OperatorTok{*} \NormalTok{depth)  }\CommentTok{# mg/L}

    \ControlFlowTok{if} \NormalTok{transfer_model }\OperatorTok{==} \StringTok{"simple-mt"}\NormalTok{:}
        \NormalTok{mass_ro }\OperatorTok{=} \NormalTok{conc_layer_aq }\OperatorTok{*} \NormalTok{runoff_mm }\OperatorTok{*} \NormalTok{cellarea()  }\CommentTok{# mg}
        \NormalTok{deltaMass_ro }\OperatorTok{=} \NormalTok{mass_ro }\OperatorTok{*} \NormalTok{delta_layer}
    \ControlFlowTok{elif} \NormalTok{transfer_model }\OperatorTok{==} \StringTok{"nu-mlm-ro"}\NormalTok{:}
        \CommentTok{# non-uniform-mixing-layer-model-runoff (nu-mlm-ro)}
        \CommentTok{# Considers a decrease in effective transfer as mixing layer depth increases}
        \CommentTok{# Adapted from Ahuja and Lehman, 1983 in @Shi2011,}
        \CommentTok{# Adaptation replaces Precip by Runoff amount.}
        \NormalTok{b }\OperatorTok{=} \DecValTok{1}  \CommentTok{# [mm] Calibration constant, 1 >= b > 0 (b-ranges appear reasonable).}
        \CommentTok{# As b decreases, mass transfer increases, model.z0 in mm}
        \NormalTok{mass_ro }\OperatorTok{=} \NormalTok{(runoff_mm }\OperatorTok{*} \NormalTok{cellarea()) }\OperatorTok{*} \NormalTok{exp(}\OperatorTok{-}\NormalTok{b }\OperatorTok{*} \NormalTok{model.z0) }\OperatorTok{*} \NormalTok{conc_layer_aq  }\CommentTok{# mg}
        \NormalTok{deltaMass_ro }\OperatorTok{=} \NormalTok{mass_ro }\OperatorTok{*} \NormalTok{delta_layer}
    \ControlFlowTok{elif} \NormalTok{transfer_model }\OperatorTok{==} \StringTok{"nu-mlm"}\NormalTok{:}
        \CommentTok{# non-uniform-mixing-layer-model (nu-mlm)}
        \CommentTok{# Original from Ahuja and Lehman, 1983 in @Shi2011}
        \NormalTok{b }\OperatorTok{=} \DecValTok{1}  \CommentTok{# [mm] Calibration constant, 1 >= b > 0 (b-ranges appear reasonable).}
        \CommentTok{# As b decreases, mass transfer increases, model.z0 in mm}
        \NormalTok{mass_ro }\OperatorTok{=} \NormalTok{(precip }\OperatorTok{*} \NormalTok{cellarea()) }\OperatorTok{*} \NormalTok{exp(}\OperatorTok{-}\NormalTok{b }\OperatorTok{*} \NormalTok{model.z0) }\OperatorTok{*} \NormalTok{conc_layer_aq  }\CommentTok{# mg}
        \NormalTok{deltaMass_ro }\OperatorTok{=} \NormalTok{mass_ro }\OperatorTok{*} \NormalTok{delta_layer}
    \ControlFlowTok{elif} \NormalTok{transfer_model }\OperatorTok{==} \StringTok{"d-mlm"}\NormalTok{:}
        \CommentTok{# distributed mixing-layer-model (d-mlm)}
        \CommentTok{# Adapted from Havis et al., 1992, and}
        \CommentTok{# taking the K_L definition for laminar flow from Bennett and Myers, 1982.}
        \NormalTok{mass_ro }\OperatorTok{=} \NormalTok{getKfilm(model, runoff_mm) }\OperatorTok{*} \NormalTok{cellarea() }\OperatorTok{*} \NormalTok{conc_layer_aq  }\CommentTok{# mg}
        \NormalTok{deltaMass_ro }\OperatorTok{=} \NormalTok{mass_ro }\OperatorTok{*} \NormalTok{delta_layer}
    \ControlFlowTok{else}\NormalTok{:}
        \BuiltInTok{print}\NormalTok{(}\StringTok{"Run-off transfer model not stated"}\NormalTok{)}
        \ControlFlowTok{return} \VariableTok{None}
        
    \ControlFlowTok{return} \NormalTok{\{}\StringTok{"mass_runoff"}\NormalTok{: mass_ro, }\StringTok{"deltaMass_runoff"}\NormalTok{: deltaMass_ro\}}
    
\end{Highlighting}
\end{Shaded}

\bibliography{LargeFiles/library.bib}


\end{document}
