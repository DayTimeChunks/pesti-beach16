\documentclass[]{article}
\usepackage{lmodern}
\usepackage{amssymb,amsmath}
\usepackage{ifxetex,ifluatex}
\usepackage{fixltx2e} % provides \textsubscript
\ifnum 0\ifxetex 1\fi\ifluatex 1\fi=0 % if pdftex
  \usepackage[T1]{fontenc}
  \usepackage[utf8]{inputenc}
\else % if luatex or xelatex
  \ifxetex
    \usepackage{mathspec}
  \else
    \usepackage{fontspec}
  \fi
  \defaultfontfeatures{Ligatures=TeX,Scale=MatchLowercase}
\fi
% use upquote if available, for straight quotes in verbatim environments
\IfFileExists{upquote.sty}{\usepackage{upquote}}{}
% use microtype if available
\IfFileExists{microtype.sty}{%
\usepackage{microtype}
\UseMicrotypeSet[protrusion]{basicmath} % disable protrusion for tt fonts
}{}
\usepackage[margin=1in]{geometry}
\usepackage{hyperref}
\hypersetup{unicode=true,
            pdftitle={BEACH Formalisms},
            pdfborder={0 0 0},
            breaklinks=true}
\urlstyle{same}  % don't use monospace font for urls
\usepackage{natbib}
\bibliographystyle{plainnat}
\usepackage{graphicx,grffile}
\makeatletter
\def\maxwidth{\ifdim\Gin@nat@width>\linewidth\linewidth\else\Gin@nat@width\fi}
\def\maxheight{\ifdim\Gin@nat@height>\textheight\textheight\else\Gin@nat@height\fi}
\makeatother
% Scale images if necessary, so that they will not overflow the page
% margins by default, and it is still possible to overwrite the defaults
% using explicit options in \includegraphics[width, height, ...]{}
\setkeys{Gin}{width=\maxwidth,height=\maxheight,keepaspectratio}
\IfFileExists{parskip.sty}{%
\usepackage{parskip}
}{% else
\setlength{\parindent}{0pt}
\setlength{\parskip}{6pt plus 2pt minus 1pt}
}
\setlength{\emergencystretch}{3em}  % prevent overfull lines
\providecommand{\tightlist}{%
  \setlength{\itemsep}{0pt}\setlength{\parskip}{0pt}}
\setcounter{secnumdepth}{0}
% Redefines (sub)paragraphs to behave more like sections
\ifx\paragraph\undefined\else
\let\oldparagraph\paragraph
\renewcommand{\paragraph}[1]{\oldparagraph{#1}\mbox{}}
\fi
\ifx\subparagraph\undefined\else
\let\oldsubparagraph\subparagraph
\renewcommand{\subparagraph}[1]{\oldsubparagraph{#1}\mbox{}}
\fi

%%% Use protect on footnotes to avoid problems with footnotes in titles
\let\rmarkdownfootnote\footnote%
\def\footnote{\protect\rmarkdownfootnote}

%%% Change title format to be more compact
\usepackage{titling}

% Create subtitle command for use in maketitle
\newcommand{\subtitle}[1]{
  \posttitle{
    \begin{center}\large#1\end{center}
    }
}

\setlength{\droptitle}{-2em}
  \title{BEACH Formalisms}
  \pretitle{\vspace{\droptitle}\centering\huge}
  \posttitle{\par}
  \author{}
  \preauthor{}\postauthor{}
  \date{}
  \predate{}\postdate{}

\usepackage{mathtools} \usepackage{natbib}

\begin{document}
\maketitle

\hypertarget{methodology}{%
\section{Methodology}\label{methodology}}

\hypertarget{hydrological-framework}{%
\subsection{Hydrological Framework}\label{hydrological-framework}}

To determine the change in soil moisture content
(\(\frac{d\theta_i}{dt}\)) in the unsaturated zone at each cell
\emph{i}, the following hydrological processes are included:

\begin{equation}
D\frac{d\theta_i}{dt} = P_i + R_i + \Delta LF_i - Ea_i - Ta_i - DP_i 
\label{eq:hydro} 
\end{equation}

where \emph{D} is depth of soil moisture simulation (mm), \(\theta\) is
soil moisture content (m\textsuperscript{3} m\textsuperscript{-3}),
\emph{dt} is model time step (day), \emph{P} is precipitation (mm),
\emph{R} is runoff (mm), \(\Delta LF\) is the difference between lateral
inflow and outflow from a given cell (mm), \emph{Ea} is the actual
evaporation (mm), \emph{Ta} actual transpiration (mm) and \emph{DP} is
the deep percolation (mm).

\hypertarget{infiltration-and-runoff}{%
\subsubsection{Infiltration and Runoff}\label{infiltration-and-runoff}}

To calculate infiltration \emph{I} and surface runoff \emph{R}, soil
moisture conditions are determined by following the SCS curve number
(BEACH-CN) method defined by the U.S. Soil Conservation Service (SCS,
1972). The method depends on permeability, land use, slope and
antecedent moisture conditions. Curve numbers are classified according
to three moisture conditions: dry (wilting point - \(CN_1\)), average
moisture (\(CN_2\)) and wet (field capacity - \(CN_3\)).

Typical curve number values for average moisture conditions (i.e.
\(CN_2\)) for various land covers, hydrologic conditions and soil types
at a 5\% slope are given in Nietsch et al. \citeyearpar{Neitsch2009}.
\(CN_2\) values are adjusted dynamically (i.e.~per time step) by crop
cover and leaf area index (LAI) and used to derive the value of \(CN_3\)
before slope adjustment:

\begin{equation}  
CN_3 = CN_2 \cdot \exp [0.00673 \cdot (100-CN_2)] 
\label{eq:CN3}  
\end{equation}

The CN2 values are then re-adjusted to account for slope differences
such that \citep{Neitsch2009}:

\begin{equation}  
CN_{2s} = \frac{CN_3 - CN_2}{3} \cdot [1-2 \cdot \exp (-13.86 \cdot slope)] + CN_2 
\label{eq:CN2s}  
\end{equation}

where \(CN_{2s}\) is the curve number for average moisture conditions
adjusted to the local slope. \(CN_1\) values accounting for slope are
then calculated as:

\begin{equation}  
CN_{1} = CN_{2s} - \frac{20 \cdot (100 - CN_{2s}}{\Big(100 - CN_{2s} + \exp [2.533-0.0636 \cdot (100 - CN_{2s})]\Big)} 
\label{eq:CN1}  
\end{equation}

Finally, adjustment to \(CN_3\) values are calculated by using the
original equation and replacing \(CN_{2s}\) by \(CN_2\).

The run-off equation is given by \citep{Neitsch2009}:

\[
    R = 
\begin{dcases}
     ~0,                                     & P \leq I_a \\
    \frac{(P-I_a)^2}{P-I_a+S},              & P > I_a 
\end{dcases}
\stepcounter{equation}\tag{\theequation}\label{eq:R}
\]

where \(I_a\) (mm) is the initial abstraction capacity of the surface
layer, which includes surface storage, interception and infiltration
prior to runoff, and typically ranges from 0.05S to 0.2S
\citep{Lim2006}. The model adopts the latter of these values as it has
provided reliable results for previous rainfall-runoff events. S is the
retention parameter after run-off (mm) given as a function of the soil
profile water content:

\begin{equation}
S = S_{max} \cdot \Big(1-\frac{SW}{(SW+ \exp[w_1-w_2 \cdot SW]  )} \Big) 
\label{eq:S}  
\end{equation}

where w1 (mm) and w2 (-) are shape coefficients, SW is the soil profile
water content excluding the amount of water held in the soil profile at
wilting point (mm) such that:

\begin{equation}
SW = max \Big[ \Big\{(\frac{D_{zl} \theta _{zl} + D_1 \theta _1}{D_{zl}+D_1} - \theta _{wp} ) \cdot (D_{zl}+D_1) \Big\},\Big\{0 \Big\} \Big]
\label{eq:SW}
\end{equation}

and \(S_{max}\) is the maximum value that the retention parameter (mm)
can take such that:

\begin{equation}
S_{max} = 254 \cdot \big(\frac{100}{CN_1} -1 \big)
\label{eq:Smax}  
\end{equation}

Calculation of \(w_1\) and \(w_2\) further assumes that:

\begin{equation}
S_3= 254 \cdot \big(\frac{100}{CN_3}-1 \big)
\label{eq:S3}
\end{equation}

and that when completely saturated \(CN\) = 99 (S = 2.54 mm) such that:

\begin{equation}
w_1 = ln \Big[\frac{FC}{(1-\frac{S_3}{S_{max}}} )-FC \Big]+w_2 \cdot FC
\label{eq:w1}
\end{equation}

\begin{equation}
w_2 = \frac{ln \Big[\frac{FC}{(1-\frac{S_3}{S_{max}}} )-FC \Big]- ln \Big[ \frac{SAT}{(1-\frac{2.54}{S_{max}}}-SAT \Big]}{SAT-FC}
\label{eq:w2}  \end{equation}

where FC is the soil profile water content at field capacity (mm), S3 is
the retention parameter (mm) corresponding to field capacity (i.e.~CN3)
and SAT is the soil profile water content at saturation (mm).

Infiltration (mm) is then given by:

\begin{equation}
I=P-R   
\label{eq:I}  
\end{equation}

After computation of the runoff equation and infiltration (and in order
to compute evapotranspiration, drainage and lateral flow), the cell
moisture content for the top layer needs to be updated. This is done by
adding infiltration (I) across the soil profile and subtracting the
fraction of water content that exceeds the saturation capacity
\(\theta_{satex,t}\) such that:

\begin{equation}
\theta_{t+1}=(\frac{\theta_t+I}{D_1} )-\theta_{satex,t}
\label{eq:thetat1}  
\end{equation}

\begin{equation}
\theta_{satex,t}=(\frac{\theta_t+I}{D_1})-\theta_{sat},  ~~~~if~~(\frac{\theta_t+I}{D_1})>\theta_{sat}
\label{eq:thetat}  
\end{equation}

The water content exceeding saturation capacity plus calculated runoff
is then added together and reported as actual runoff. The discharge at
the outlet is calculated here as a function of time and spatial
propagation (i.e.~LDD, Runoff).

\hypertarget{deep-percolation}{%
\subsubsection{Deep Percolation}\label{deep-percolation}}

Deep percolation (DP) is assumed to be negligible at moisture levels
below field capacity. Above this moisture level percolation is given by
Raes \textit{et al.}, {[}\citet{Raes:2002}\}:

\begin{equation}
DP_z = D_z \tau_z (\theta_{sat, z} - \theta_{fc, z}) \frac{e^{\theta_z-\theta_{fc,z}}-1}{e^{\theta_{sat, z}-\theta_{fc,z}}-1},~~if~ \theta_z > \theta_{fc, z} 
\label{eq:DP}  
\end{equation}

where \(D_z\) is the soil profile depth of layer \(z\) and \(\tau\) is a
dimensionless drainage characteristic given by:

\begin{equation}
\tau = 0.0866 \cdot e^{0.8063 \cdot log_{10}(K_{sat})}, ~~0< \tau \leq 1
\label{eq:tau}  
\end{equation}

where \(K_{sat}\) is the saturated hydraulic conductivity.

\hypertarget{lateralsubsurface-flow}{%
\subsubsection{Lateral/subsurface flow}\label{lateralsubsurface-flow}}

Later flow occurs when the soil moisture content exceeds the field
capacity and is represented as the difference between inflow and outflow
at a cell \(j\) for each soil layer \(z\) according to Manfreda
\textit{et al.}, \citep{Manfreda2005}:

\begin{equation}
\Delta LF_{j(t)} = \Big( \frac{W_{j} \sum^{N(t)}_{i=1}max[c_{z}(SW_{i}-SW_{fc,i}),~0] }{ \sum^{N(t)}_{i=1} W_{i} } \Big) - max[c_z(SW_j-SW_{fc,j}),~0]
\label{eq:LF}  
\end{equation}

were \(c_z\) is the subsurface flow coefficient (\(\approx\) 0.25
\(d^{-1}\), if \(dt\) \(\leq\) \(1~d\), \citep{Manfreda2005}), \(SW\)
and \(SW_{fc}\) are the soil water content at time \(t\) and at field
capacity for the soil profile {[}mm{]}, respectively. \(N(t)\) is the
number of cells \(i\) upstream exceeding field capacity and at steepest
slope to cell \(j\). The topographical wetness index (TWI) \(W\),
introduced by \cite{Beven1979}, describes the tendency for water to
accumulate spatially in regions characterized by a relatively low local
slope and a large upstream drainage area. \(W_j\) is the TWI at cell
\(j\) and \(W_i\) the cumulative TWI from upstream cells \(i\). The TWI
is given by:

\begin{equation}
TWI =ln\Big(\frac{a}{tan \beta} \Big)
\label{eq:Wz}  
\end{equation}

were \(a\) is the drainage area per unit contour length (defined by the
Local Drain Direction (LDD) network and cell area) and tan \(\beta\),
the local slope in radians derived from the Digital Elevation Model
(DEM).

\hypertarget{evapotranspiration}{%
\subsubsection{Evapotranspiration}\label{evapotranspiration}}

To account for evapotranspiration processes the FAO56 reference
evaporation rate, \(ET_0\) (mm), has been considered and adjusted
dynamically according to crop and climate-specific factors. The approach
assumes a dual crop coefficient approach appropriate for daily time-step
calculations \citep{Allen1998} and made up of a basal crop coefficient
(\(K_{cb}\)) and a soil water evaporation coefficient (\(K_e\)).

Potential evapotranspiration \((ET_p)\) is then given by

\begin{equation}
ET_p=K_c \cdot ET_0
\label{eq:ETp}  
\end{equation}

\begin{equation}
K_c = K_{cb} + K_e
\label{eq:Kc}  
\end{equation}

where \(K_{cb}\) varies according to crop-specific development stage. In
cases where the mean value for daily relative humidity during the mid-
or late-season growth stage (\(RH_{min}\)\%) differs from 45\% or where
wind speed varies by more than 2 m/s the \(K_{cb}\) values for mid- and
late-season must be adjusted according to:

\begin{equation}
K_{cb}=K_{cb_{mid/end}} + \Big[ 0.04(U_2-2)-0.004(RH_{min}-45) \Big] \Big( \frac{h_{crop}}{3} \Big) ^{0.3}
\label{eq:Kcb}  
\end{equation}

\begin{equation}
K_e= K_{cmax}-K_{cb}
\label{eq:Ke}  
\end{equation}

where \(K_{cb_{mid/end}}\) represent the reference values for sub-humid
climate and moderate wind speeds \citep[see][]{Allen1998}. \(U2\) is the
wind speed at a height of 2 meters (m/s), \(RH_{min}\) is the minimum
relative humidity (\%) and \(h_crop\) is crop height.

\(K_e\) is the soil evaporation coefficient and \(K_{cmax}\) represents
an upper limit to evapotranspiration from cropped surfaces, typically
ranges between 1.05 to 1.30 (-) and given by \citep{Sheikh2009}:

\begin{equation}
K_{cmax}=max \Big[ \Big\{ K_{cb}+0.05 \Big\} , \Big\{ 1.2 + [0.04(U_2-2)-0.004(RH_{min}-45)] \cdot (\frac{h}{3})^{0.3} \Big\} \Big]
\label{eq:Kcmax}  
\end{equation}

When available soil moisture is limited potential evapotranspiration is
reduced (i.e., \(ET_a < ET_p\)).

\hypertarget{transpiration}{%
\paragraph{Transpiration}\label{transpiration}}

To account for potential transpiration processes, water uptake by roots
is considered and regulated by atmospheric demand and soil water
content. When there is sufficient water in the soil, potential
transpiration (\(T_p\)) equals atmospheric demand
\citep{Sheikh2009}\footnote{Error in Sheikh2009, where $K_{c}f$ should be $K_{cb}$}:

\begin{equation}
T_p=K_{cb} \cdot ET_0
\label{eq:Tp}  
\end{equation}

Potential transpiration is further subject to root water uptake capacity
where the maximum daily uptake \(T_{p(z)}\) (mm) at each layer \(z\) is
given by \citep{Prasad1988}:

\begin{equation}
T_{p(z)} = 2 \Big( 1- \frac{RD_{z,0.5} }{RD} \Big) \Big( \frac{RD_z}{RD} \Big) T_p
\label{eq:Tpz}  
\end{equation}

where \(RD\) (m) is the total or individual soil layer's rooting depth
and \(RD_{z, 0.5}\) is the soil depth at the middle of the root
extension for layer \(z\).

When soil water is insufficient to meet atmospheric demand, actual
transpiration is lower than potential transpiration and given by
\citep{Sheikh2009}:

\begin{equation}
T_{a(z)} = K_s \cdot T_p
\label{eq:Taz}  
\end{equation}

\begin{equation}
K_s= max \Big[ 0, min(1, \frac{\theta_t - \theta_{wp} }{ \theta_c - \theta_{wp} }) \Big]
\label{eq:Ks}  
\end{equation}

\begin{equation}
\theta_c = \theta_{wp} + (1 - p)(\theta_{fc}- \theta_{wp}) 
\label{eq:theta_c}  
\end{equation}

\begin{equation}
p = p_{tab} + 0.04(5-ET_p)
\label{eq:p}  
\end{equation}

where \(K_s\) is a transpiration reduction parameter (0-1), which
depends on soil water content, \(\theta_t\) \((m^3/m^3)\) and the
critical soil moisture content \(\theta_c\) \((m^3/m^3)\) that defines
the transition between unstressed and stressed transpiration rate. \(p\)
(-) is the fraction of total depletable soil water and \(p_{tab}\) the
depletion factor (-) for \(ET_p \approx 5\) \(mm/d\) \citep[Table no.
22]{Allen1998}.

\hypertarget{evaporation}{%
\paragraph{Evaporation}\label{evaporation}}

Evaporation is considered only on bare surfaces and assumed to be
negligible under plant cover and regulated by atmospheric deman along
the first \(\approx\) 0.15 m of soil \citep{Sheikh2009}. Considering the
difference between actual (\(E_a\), mm/d) and potential evaporation
(\(E_p\), mm/d) \citep{Allen1998}:

\begin{equation}
E_p=K_e \cdot ET_0
\label{eq:Ep}  
\end{equation}

\begin{equation}
E_a=K_r \cdot E_p
\label{eq:Ea}  
\end{equation}

where \(K_r\) is an evaporation reduction coefficient (-) given by:

\begin{equation}
K_r = \frac{ \theta_t - \theta_{dr} }{ \theta_{fc} - \theta_{dr} }
\label{eq:Kr}  
\end{equation}

where \(\theta_t\) is soil moisture (\(m^3/m^3\)) and \(\theta_{dr}\) is
the moisture (\(m^3/m^3\)) of air-dry
soil\footnote{Note: In model, $\theta_{dr} = 0.33 \cdot \theta_{wp}$ (Sheikh et al., 2009)}.

\hypertarget{crop-growth}{%
\subsubsection{Crop Growth}\label{crop-growth}}

\ldots{}to document

\hypertarget{pesticide-transfer}{%
\subsection{Pesticide Transfer}\label{pesticide-transfer}}

The concentration of pesticide within the soil system (\(mg/L\)) along
each layer \(z\) is given as a function of its partitioning across the
soil's gaseous, aqueous and adsorbed phases by:

\begin{equation}
C_{tot_z} = \theta_{gas_z}(t)C_{gas_z}(t) + \theta_z(t)C_{aq_z}(t) + \rho_{b_z}(t)C_{ads_z}
\label{eq:conc_tot}  
\end{equation}

where \(\theta_{gas_z}(t) = \theta_{sat_z}(t) - \theta_{z}(t)\),
\(C_{gas_z}\) is the concentration in the gas phase (\(mg/L\)),
\(C_{aq_z}\) the concentration in aqueous phase (\(mg/L\)) and
\(C_{ads_z}\) the concentration in adsorbed phase (\(mg/g\)) with the
bulk soil density \(\rho_{b_z}\) in \(g/L\).

Converting to mass base don the cell area (\(A_i\), \(m^2\)) and model
layer depth (\(D_z\), \(mm\)):

\begin{equation}
M_{tot_z} = A_iD_z\theta_{gas_z}(t)C_{gas_z}(t) + A_iD_z\theta_z(t)C_{aq_z}(t) + A_iD_z\rho_{b_z}(t)C_{ads_z}
\label{eq:mass_tot}  
\end{equation}

simplifying,

\begin{equation}
M_{tot_z} = V_{gas_z}(t)C_{gas_z}(t) + V_{aq}(t)C_{aq_z}(t) + V_{ads}(t)C_{ads_z}
\label{eq:mass_tot_simple}  
\end{equation}

and substituting phase concentrations for their equivalent in \(C_{aq}\)
according to eq. \ref{eq:kd} and eq. \ref{eq:henry} yields,

\begin{equation}
M_{tot_z} = V_{gas_z}K_H C_{aq_z}(t) + V_{aq}(t)C_{aq_z}(t) + V_{ads}(t)K_dC_{aq_z}(t)
\label{eq:mass_tot_sub}  
\end{equation}

Solving for \(C_{aq}\),

\begin{equation}
C_{aq} = \frac{M_{tot_z} }{  V_{gas_z}(t)K_H + V_{aq}(t) + V_{ads}(t)K_d }
\label{eq:mass_tot_conc_aq1}  
\end{equation}

\begin{equation}
C_{aq} = \frac{M_{tot_z} }{ A_iD_z \Big( \theta_{gas_z}(t)K_H + \theta_{aq}(t) + \rho_{b_z}(t)K_d \Big)}
\label{eq:mass_tot_conc_aq2}  
\end{equation}

and substituting the retardation factor from eq. \ref{eq:retard_linear},

\begin{equation}
C_{aq} = \frac{M_{tot_z} }{ A_iD_z \Big( \theta_{gas_z}(t)K_H + \theta_{aq}(t)R_z(t) \Big)}
\label{eq:mass_tot_conc_aq2}  
\end{equation}

\hypertarget{sorption}{%
\subsubsection{Sorption}\label{sorption}}

\hypertarget{linear-sorption}{%
\paragraph{Linear sorption}\label{linear-sorption}}

The partition of pesticide concentrations into the dissolved \(C_{aq}\)
(\(mg/m^3\)) and adsorbed phases \(C_{ads}\) (mg/Kg) is determined by
the pesticide dissociation coefficient \(K_d\) (L/Kg):

\begin{equation}
K_d = \frac{ C_{ads} }{ C_{aq}  }
\label{eq:kd}
\end{equation}

\begin{equation}
K_d = K_{oc} \cdot f_{oc}
\label{eq:kd}
\end{equation}

where \(K_{oc}\) is the pesticide octanol partition coefficient (mL/g)
and \(f_{oc}\) is the fraction of organic carbon (-) in soil.

The dissolved concentration (\(mg/L\)) is given by \citep{Whelan1987}:

\begin{equation}
C_{aq_z}(t) = \frac{ M_{p_z}(t) }{\theta_z(t) \cdot R_z{(t)} \cdot D_{z} }
\label{eq:conc_aq}
\end{equation}

and the adsorbed contaminant concentration (\(mg/Kg~dry~soil\)):

\begin{equation}
C_{ads_z}(t) = \frac{ M_{p_z}(t) K_d}{\theta_z(t) \cdot R_z{(t)} \cdot D_{z} }
\label{eq:conc_ads}
\end{equation}

where the pesticide mass \(M_p\) (\(mg/m^2\)) is assumed to be perfectly
mixed within the soil profile's water content \(\theta\) (-) of a soil
layer \(z\) with depth \(D\) (mm).

The retardation factor \(R_z\) (-) is given by:

\begin{equation}
R_z(t) = 1 + \frac{ \rho_{b_z}(t) \cdot K_d }{ \theta_z(t) }
\label{eq:retard_linear}
\end{equation}

where \(\rho_{b_z}\) is bulk density (Kg m\(^{-3}\)) and pesticide
dissociation coefficient \(K_d\) (mL/g).

\hypertarget{volatilization}{%
\subsubsection{Volatilization}\label{volatilization}}

Pesticide volatilization follows Leistra \textit{et al.}
\citeyearpar{Leistra2001}, where a boundary air layer is conceptualized
through which pesticide difusses before escaping into the atmosphere.
The thickness (\(d_a\), \(m\)) of this layer, assumed to be equivalent
to the topmost soil layer's thickness or mixing layer, regulates the
transport resistance (\(r_a\), \(d/m\)) such that:

\begin{equation}
r_a(t) = \frac{ d_a }{ D_a(t) }
\label{eq:resistance_air}
\end{equation}

where \(D_a\) (\(m^2/d\)) is the diffusion coefficient in air for
Metolachlor at the observed environmental temperature and adjusted
relative to the reference diffusion coefficient (\(D_{a,r}\), \(m^2/d\))
as:

\begin{equation}
D_a(t) = \Big(\frac{ T(t) }{ T_r } \Big)^{1.75} D_{a,r}
\label{eq:Da}
\end{equation}

where \(T\) and \(T_r\) are the environmental temperature at time \(t\)
and at the reference temperature at 293.15\(^\circ\)K, respectively.

The total volatilization is given by the flux across the air layer
boundary (\(J_{v,air}\)) and the flux across the topmost soil layer
(\(J_{v,soil}\)) such that:

\begin{equation}
J_{v,air}(t) = - \frac{ C_{gas,top}(t) - C_{air}(t) }{ r_a }
\label{eq:Jva}
\end{equation}

\begin{equation}
J_{v,soil}(t) = - \frac{ C_{gas,z_0}(t) - C_{gas,top}(t) }{ r_s }
\label{eq:Jvs}
\end{equation}

where \(C_{gas,top}\) (\(mg/m^3\)) is the concentration in gas phase at
the soil surface, \(C_{air}\) (\(mg/m^3\)) the concentration in air,
\(C_{gas,z_0}\) (\(mg/m^3\)) the concentration in gas phase at the
center of the uppermost soil layer and \(r_s\) (\(d/m\)) the diffusion
resistance across the topmost soil layer and given by:

\begin{equation}
r_s(t)= \frac{ 0.5 D_z }{ D_{rdiff,g}(t) }
\label{eq:resistance_soil}
\end{equation}

To calculate the relative diffusion (\(D_{rdiff,gas}\), \(m^2/d\)) the
model provides two options. Under option 1 \citep{Millington1960},

\begin{equation}
D_{rdiff,gas} = \frac{ D_a(t) \Big(\theta_{gas_z}(t)\Big)^a}{ \Big(\theta_z(t) \Big)^b }
\label{eq:d_rdiff1}
\end{equation}

where Jin and Jury \citeyearpar{Jin1996} recommend that \(a=2\) and
\(b=2/3\). Under option 2 \citep{Currie1960},

\begin{equation}
D_{rdiff,gas} = D_a(t) \Big(a\Big) \Big( \theta_{gas_z}(t)  \Big)^b
\label{eq:d_rdiff2}
\end{equation}

where Bakker \textit{et al.} \citeyearpar{Bakker1987} recommend
\(a=2.5\) and \(b=3\) for moderately aggregated plough layers of loamy
soils and humic sandy soils \citep{Leistra2001}.

Finally, it is assumed that flux across both layer boundaries is
equivalent (\(J_{v,soil} = J_{v,air}\)) \citep{Leistra2001}. Considering
pesticide concentration in air to be negligble (\(C_{air} = 0\)), the
concentration at the soil surface is:

\begin{equation}
C_{gas,top}(t) = \frac{r_a}{(r_a + r_s)} C_{gas,z_{0}(t)}
\label{eq:conc_gas_top}
\end{equation}

The gas concentration in the soil layer is related to the dimensionless
Henry constant (\(K_H\)), where:

\begin{equation}
C_{gas,z_0}(t) = C_{aq,z_0}(t) K_H  
\label{eq:henry}
\end{equation}

Substituting eq. \ref{eq:conc_gas_top} into eq. \ref{eq:Jva} yields the
mass flux lost to the atmosphere (\(mg/m^2d\)):

\begin{equation}
J_{v,air} = - \frac{C_{gas,z_0}}{(r_a + r_s)}
\label{eq:Jva_final}
\end{equation}

\hypertarget{run-off-loss}{%
\subsubsection{Run-off Loss}\label{run-off-loss}}

\hypertarget{section}{%
\paragraph{\texorpdfstring{\textit{Non-uniform mixing-layer-model-runoff (nu-mlm-ro)}}{}}\label{section}}

Multiple models are available to simulate mass transfer to overland
flow. The first of these models, the
\textit{Non-uniform mixing-layer-model-runoff} (nu-mlm-ro) is adapted
from Ahuja and Lehman, 1983 \citep[see][eq. 1 and p.~1217]{Shi2011} and
given by:

\begin{equation}
\frac{\partial (EDI \theta C_m)}{\partial t} = -RO \beta C_m
\label{eq:nu-mlm-ro}
\end{equation}

\begin{equation}
\beta = e^{(-bz)}
\label{eq:beta-nu-mlm}
\end{equation}

where the Effective Depth of Interaction (EDI) refers to the mixing
layer depth (mm), \(\theta\) is soil moisture \((m^3/m^3)\), RO is
run-off (mm) and \(C_m\) is concentration in the mixing layer
(mg/mm\(^3\)). The paramater \(b\) is a calibration constant (assuming,
\(1 \ge b > 0\)) and where \(z\) is the depth of the simulated top-soil
layer. In this model, \(\beta\) accounts for an exponential decrease in
the ability of overland flow to mix with soil water as depth increases.

\hypertarget{section-1}{%
\paragraph{\texorpdfstring{\textit{Non-uniform mixing-layer (nu-mlm)}}{}}\label{section-1}}

The adaptation above of the original model by Ahuja and Lehman replaces
precipitation by \(RO\), and thus considers the ability of rainfall
water to mix with soil water instead. To test the original formulation,
this second model is also made available as
\textit{Non-uniform mixing layer model} (nu-mlm).

\hypertarget{section-2}{%
\paragraph{\texorpdfstring{\textit{Distributed mixing-layer model (d-mlm)}}{}}\label{section-2}}

The \textit{Distributed mixing-layer model} considered is adapted from
Havis \citeyearpar{Havis1992}, and mass transfer to overland flow based
on a mass flux coefficient (\(K_L\), mm/day). The change in mass in the
overland flow (and consequently in the mixing layer) is given by:

\begin{equation}
\frac{\partial (h_{runoff} C_{runoff}) }{\partial t} + \frac{\partial(qC_{runoff}) }{\partial x} = K_L (C_m - C_{runoff})
\label{eq:d-mlm-of}
\end{equation}

where \(h_{runoff}\) is overland flow height (mm), \(q\) is overland
flow discharge rate per unit width (mm\(^2\)/day).

Instead of considering \(K_L\) as calibration parameter, the model
defines \(K_L\) (mm/day) as the ratio of the mass flux (\(\varrho\),
mg/mm\(^2\) day) from the soil surface to overland flow and the solute
concentration difference between overland flow and the surface soil
(i.e., \(K_L = \varrho /(C_m - C_{runoff})\)). Note that due to small
concentations in overland flow, it may be assumed that
\(C_{runoff} \approx 0\). For laminar flow, based on Bennett and Myers
\citeyearpar{Bennet1982}, \(K_L\) is given by:

\begin{equation}
K_L = 0.664 \frac{D_w}{L}Re^{1/2}S_c^{1/3}
\label{eq:K_L}
\end{equation}

\begin{equation}
Re = \frac{\rho v L}{\mu}
\label{eq:Re}
\end{equation}

\begin{equation}
S_c = \frac{ \mu}{\rho D_w}
\label{eq:S_c}
\end{equation}

where the \(D_w\) is the solute diffusivity (cm\(^2\)/s), \(Re\) is the
Reynolds number (-) and \(S_c\) is the Schmidt number (-). Parameters
include the cell length (L, mm), the dynamic viscocity of water (\(\mu\)
at 25 \(^\circ C\), 8.9e-03 g/cm sec), the soil bulk density (\(\rho\),
g/cm\(^3\)) and the runoff velocity (\(v\), mm/day) or amount of runoff
generated in each cell per day.

\hypertarget{vertical-mass-flux-i.e.-leaching}{%
\subsubsection{Vertical mass flux (i.e.,
leaching)}\label{vertical-mass-flux-i.e.-leaching}}

Vertical flux can be computed differently across soil layers. Under the
first approach, and only for the uppermost layer, the model follows
McGrath \citeyearpar{Mcgrath2008}:

\begin{equation}
C_{z_0,aq}(t+1) = C_{z_0,aq}(t) exp \Big( \frac{ -P(t) }{ \theta_{z_0}(t) \cdot R_{z_0}(t) \cdot D_{z_0} } \Big) 
\label{eq:conc_mcgrath}
\end{equation}

The mass leached (\(mg\)) is thus given by:

\begin{equation}
M_{z_0,lch}(t) =  D_{z_0} \cdot A_i  \Big(\theta_{z_0}(t)C_{z_0,aq}(t)- \theta_{z_0}(t+1) C_{z_0,aq}(t+1) \Big) 
\label{eq:leached_mcgrath}
\end{equation}

where \(A\) is the area (\(m^2\)) for each cell \(i\).

Under the second approach, available on all layers, mass leached is
proportional to the aqueous concentration in percolated water such that:

\begin{equation}
M_{z,lch}(t) = DP_z(t) \cdot C_{z,aq }(t) \cdot A_i
\label{eq:leached_prop}
\end{equation}

\hypertarget{lateral-mass-flux}{%
\subsubsection{Lateral mass flux}\label{lateral-mass-flux}}

Later mass flux is proportional to the aqueous concentration in lateral
water flow. The net mass flux is given by adapting eq. \ref{eq:LF}:

\begin{equation}
\Delta LMF_{i,z} = \Big( \frac{W_{i,z} \sum^{N(t)}_{j=1}max[c_{j,z}(s_{j,z}-s_{fc,j,z}),~0]}{ \sum^{N(t)}_{j=1} W_{j,z} } \cdot C_{j,z,aq} \Big) - max[ c_z(s_z-s_{fc,z}),~0]  \cdot C_{z,aq}
\label{eq:LMF}  
\end{equation}

\hypertarget{degradation}{%
\subsection{Degradation}\label{degradation}}

\hypertarget{photodegradation-isotopes}{%
\subsubsection{Photodegradation
(isotopes!!!)}\label{photodegradation-isotopes}}

\ldots{}to formalize and document.

\hypertarget{biodegradation-isotopes}{%
\subsubsection{Biodegradation
(isotopes!!!)}\label{biodegradation-isotopes}}

Biodegradation in the soil is assumed to occur only on the dissolved and
sorbed fraction on equilibrium sites. Assuming a first-order rate law:

\begin{equation} 
\frac{dM_{p(i,k)}}{dt} = -k_{b,L}(V_{L(t,i,k)}C_{L(t,i,k)})-k_{b,eq}(M_{s(i,k)}X_{eq(t,i,k)})
\label{eq:dMdt}
\end{equation}

where liquid phase in the pore volume
\(V_L = A_i \cdot D_k\cdot \theta_{(t,i,k)}\) and \(k_{b}\) (d\(^{-1}\))
is the biodegradation rate constant for the liquid (\(L\)) and
equilibrium-sorbed sites (\(eq\)), respectively.

Alternatively equation \eqref{dMdt} can subdivided and written in
integrated form:

\begin{equation}
M_{L(t+1,i,k)} = (V_{L(t,i,k)}C_{L(t,i,k)})e^{-k_{b} \Delta t} 
\label{eq:ML_t1}
\end{equation}

\begin{equation}
M_{eq(t+1,i,k)} = (M_{s(i,k)}X_{eq(t,i,k)})e^{-k_{b} \Delta t}
\label{eq:Meq_t1}
\end{equation}

where \(M_{L}\) and \(M_{eq}\) are the contamimant masses in liquid
phase and equilibrium-sorbed sites,
respectively.\footnote{A mass balance check should compute the equivalence of equation \eqref{dMdt} and $\Delta M_{L,eq}(t)$, equation \eqref{DeltaM}.}
Updating for total contaminant mass using equations \eqref{MassNe_t1},
\eqref{ML_t1} and \eqref{Meq_t1}:

\begin{equation}
M_{p(t+1,i,k)} = M_{L(t+1,i,k)} + M_{eq(t+1,i,k)} + M_{s(i,k)}X_{ne(t+1,i,k)}
\label{eq:totmass_t1}
\end{equation}

Before proceeding onto the following time step in the simulation, the
mass in liquid phase from equation \eqref{ML_t1}, after mass transport
processes, is used to compute a new iterative loop (i.e.~equations
\eqref{iterStep1} and \eqref{iterStep2}). Namely,

\begin{equation}
C_{L(i,k)} = \frac{M_{L(t+1,i,k)}}{A_i \cdot D_k \cdot \theta_{(t,i,k)}} 
\label{eq:C_Lnew}
\end{equation}

The biodegradation rate constant \(k_{b(L,eq)}\) is given by:

\begin{equation}
k_b = \frac{ln(2)}{t_\frac{1}{2}} 
\label{eq:k_b}
\end{equation}

where \(t_\frac{1}{2}\) is the half-life (d) of the contaminant
considered. Adapting the reference half-life to temperature and moisture
changes following Dairon (cite: Dairon , 2015) the half-life becomes:

\begin{equation}
t_\frac{1}{2}=t_\frac{1}{2}^{ref}\cdot F_T \cdot F_\theta
\label{eq:DT50} 
\end{equation}

where \(t_\frac{1}{2}^{ref}\) is the half-life (days) at the reference
moisture and temperature. \(F_T\) and \(F_\theta\) are factor changes in
degradation rates associated to temperature and moisture conditions,
respectively. To account for the influence of water content across a
range of saturation conditions \(F_\theta\) is given by (cite: Schroll
et al, 2006):

\begin{equation} 
F_\theta = 
\end{equation}

\begin{equation} 
F_T =
\end{equation}

where T is the temperature in soil layer k and time step t. \(T_{ref}\)
is the reference temperature at which the experimental half-life is
reported (i.e.~typically 20 \(^{\circ}\)C). \(E_a\) is the activation
energy generally equal to 54,000 (KJ mol\(^{-1}\)) and R is the gas
constant 8.314 (J mol\(^{-1}\) K\(^{-1}\)).

\hypertarget{soil-temperature}{%
\subsection{Soil Temperature}\label{soil-temperature}}

Temperture is an important parameter regulating chemical and biological
processes. To simulate its evolution across time and space this model
follows \citep{Neitsch2009}, where the average daily soil temperature
(\(^{\circ}\)C) at the center of each soil layer \(k\) is given by:

\begin{equation}
T_{soil, k, t} = l \cdot T_{soil, k, t-1} + (1.0 - l) \cdot \big[df \cdot [\overline{T}_{AAir} - T_{ssurf}]+T_{ssurf}]
\label{eq:tempSoil} 
\end{equation}

where \(l\) is a lag coefficient (0 - 1.0) regulating the influence of
the previous day's temperature, the subscript \(t-1\), refers to the
previous day, \(\overline{T}_{AAir}\) is the average annual air
temperature and \(T_{ssurf}\) is the soil surface temperature at time
\(t\). The depth factor \(df\) is given by,

\begin{equation}
df = \frac{zd}{zd+exp(-0.867-2.078 \cdot zd)}
\label{eq:df} 
\end{equation}

where \(zd\) is the ratio of the depth at the center of the soil layer
\(D_{k_{mid}}\) (mm) to the damping depth \(dd\) (mm) such that:
\begin{equation}
zd = \frac{D_{k_{mid}}}{dd}
\label{eq:zd} 
\end{equation}

The damping depth \(dd\) is a function of the maximum damping depth
\(dd_{max}\) (mm) and a soil water scaling factor \(\varphi\) (-).

\begin{equation}
dd = dd_{max} \cdot exp \Big[ln\Big(\frac{500}{dd_{max}} \Big) \cdot \Big(\frac{1-\varphi}{1+\varphi}\Big)^2\Big]
\label{eq:dd} 
\end{equation}

\begin{equation} 
dd_{max} = 1000 + \frac{2500\rho_b}{\rho_b+686 \cdot exp(-5.63\rho_b)}
\label{eq:ddmax} 
\end{equation}

\begin{equation} 
\varphi = \frac{SW}{(0.356-0.144\rho_b) \cdot D_{k}}
\label{eq:varphi} 
\end{equation}

where \(\rho_b\) is the soil bulk density (mg m\(^{-3}\)) and SW is
water content (mm \(H_2O\)) in the soil profile \(D_k\) (mm).

The soil surface temperature \(T_{ssurf}\) in eq. \refeq{tempSoil}, is a
fucntion of the previous day's soil temperature, amount of ground cover
captured by a crop factor \(bcv\) and the temperature of bare soil
\(T_{bare}\), such that:

\begin{equation} 
T_{ssurf} = bcv \cdot  T_{soil, k, t-1} + (1-bcv) \cdot T_{bare}
\label{eq:Tssurf} 
\end{equation}

\begin{equation} 
bcv = \frac{CV}{(CV+exp(7.563-1.297 \cdot 10^{-4} \cdot CV))}
\label{eq:Tssurf} 
\end{equation}

\begin{equation} 
    CV = 
\begin{dcases}
     ~~~~~0~,                                     & f \leq 0 \\
    \frac{lnf}{-5\cdot10^{-5}},              & f > 0 
\end{dcases}
\end{equation}

\begin{equation} 
f=1-exp(-\mu*LAI)
\label{eq:f} 
\end{equation}

where \(\mu\) is light-use efficiency of the crop (kg ha\(^{-1}\)
m\(^2\) MJ\(^{-1}\))

\renewcommand\refname{References}
\bibliography{library.bib}


\end{document}
