\documentclass[]{article}
\usepackage{lmodern}
\usepackage{amssymb,amsmath}
\usepackage{ifxetex,ifluatex}
\usepackage{fixltx2e} % provides \textsubscript
\ifnum 0\ifxetex 1\fi\ifluatex 1\fi=0 % if pdftex
  \usepackage[T1]{fontenc}
  \usepackage[utf8]{inputenc}
\else % if luatex or xelatex
  \ifxetex
    \usepackage{mathspec}
  \else
    \usepackage{fontspec}
  \fi
  \defaultfontfeatures{Ligatures=TeX,Scale=MatchLowercase}
\fi
% use upquote if available, for straight quotes in verbatim environments
\IfFileExists{upquote.sty}{\usepackage{upquote}}{}
% use microtype if available
\IfFileExists{microtype.sty}{%
\usepackage{microtype}
\UseMicrotypeSet[protrusion]{basicmath} % disable protrusion for tt fonts
}{}
\usepackage[margin=1in]{geometry}
\usepackage{hyperref}
\hypersetup{unicode=true,
            pdftitle={Isotope Tracking},
            pdfborder={0 0 0},
            breaklinks=true}
\urlstyle{same}  % don't use monospace font for urls
\usepackage{natbib}
\bibliographystyle{plainnat}
\usepackage{color}
\usepackage{fancyvrb}
\newcommand{\VerbBar}{|}
\newcommand{\VERB}{\Verb[commandchars=\\\{\}]}
\DefineVerbatimEnvironment{Highlighting}{Verbatim}{commandchars=\\\{\}}
% Add ',fontsize=\small' for more characters per line
\usepackage{framed}
\definecolor{shadecolor}{RGB}{248,248,248}
\newenvironment{Shaded}{\begin{snugshade}}{\end{snugshade}}
\newcommand{\KeywordTok}[1]{\textcolor[rgb]{0.13,0.29,0.53}{\textbf{{#1}}}}
\newcommand{\DataTypeTok}[1]{\textcolor[rgb]{0.13,0.29,0.53}{{#1}}}
\newcommand{\DecValTok}[1]{\textcolor[rgb]{0.00,0.00,0.81}{{#1}}}
\newcommand{\BaseNTok}[1]{\textcolor[rgb]{0.00,0.00,0.81}{{#1}}}
\newcommand{\FloatTok}[1]{\textcolor[rgb]{0.00,0.00,0.81}{{#1}}}
\newcommand{\ConstantTok}[1]{\textcolor[rgb]{0.00,0.00,0.00}{{#1}}}
\newcommand{\CharTok}[1]{\textcolor[rgb]{0.31,0.60,0.02}{{#1}}}
\newcommand{\SpecialCharTok}[1]{\textcolor[rgb]{0.00,0.00,0.00}{{#1}}}
\newcommand{\StringTok}[1]{\textcolor[rgb]{0.31,0.60,0.02}{{#1}}}
\newcommand{\VerbatimStringTok}[1]{\textcolor[rgb]{0.31,0.60,0.02}{{#1}}}
\newcommand{\SpecialStringTok}[1]{\textcolor[rgb]{0.31,0.60,0.02}{{#1}}}
\newcommand{\ImportTok}[1]{{#1}}
\newcommand{\CommentTok}[1]{\textcolor[rgb]{0.56,0.35,0.01}{\textit{{#1}}}}
\newcommand{\DocumentationTok}[1]{\textcolor[rgb]{0.56,0.35,0.01}{\textbf{\textit{{#1}}}}}
\newcommand{\AnnotationTok}[1]{\textcolor[rgb]{0.56,0.35,0.01}{\textbf{\textit{{#1}}}}}
\newcommand{\CommentVarTok}[1]{\textcolor[rgb]{0.56,0.35,0.01}{\textbf{\textit{{#1}}}}}
\newcommand{\OtherTok}[1]{\textcolor[rgb]{0.56,0.35,0.01}{{#1}}}
\newcommand{\FunctionTok}[1]{\textcolor[rgb]{0.00,0.00,0.00}{{#1}}}
\newcommand{\VariableTok}[1]{\textcolor[rgb]{0.00,0.00,0.00}{{#1}}}
\newcommand{\ControlFlowTok}[1]{\textcolor[rgb]{0.13,0.29,0.53}{\textbf{{#1}}}}
\newcommand{\OperatorTok}[1]{\textcolor[rgb]{0.81,0.36,0.00}{\textbf{{#1}}}}
\newcommand{\BuiltInTok}[1]{{#1}}
\newcommand{\ExtensionTok}[1]{{#1}}
\newcommand{\PreprocessorTok}[1]{\textcolor[rgb]{0.56,0.35,0.01}{\textit{{#1}}}}
\newcommand{\AttributeTok}[1]{\textcolor[rgb]{0.77,0.63,0.00}{{#1}}}
\newcommand{\RegionMarkerTok}[1]{{#1}}
\newcommand{\InformationTok}[1]{\textcolor[rgb]{0.56,0.35,0.01}{\textbf{\textit{{#1}}}}}
\newcommand{\WarningTok}[1]{\textcolor[rgb]{0.56,0.35,0.01}{\textbf{\textit{{#1}}}}}
\newcommand{\AlertTok}[1]{\textcolor[rgb]{0.94,0.16,0.16}{{#1}}}
\newcommand{\ErrorTok}[1]{\textcolor[rgb]{0.64,0.00,0.00}{\textbf{{#1}}}}
\newcommand{\NormalTok}[1]{{#1}}
\usepackage{graphicx,grffile}
\makeatletter
\def\maxwidth{\ifdim\Gin@nat@width>\linewidth\linewidth\else\Gin@nat@width\fi}
\def\maxheight{\ifdim\Gin@nat@height>\textheight\textheight\else\Gin@nat@height\fi}
\makeatother
% Scale images if necessary, so that they will not overflow the page
% margins by default, and it is still possible to overwrite the defaults
% using explicit options in \includegraphics[width, height, ...]{}
\setkeys{Gin}{width=\maxwidth,height=\maxheight,keepaspectratio}
\IfFileExists{parskip.sty}{%
\usepackage{parskip}
}{% else
\setlength{\parindent}{0pt}
\setlength{\parskip}{6pt plus 2pt minus 1pt}
}
\setlength{\emergencystretch}{3em}  % prevent overfull lines
\providecommand{\tightlist}{%
  \setlength{\itemsep}{0pt}\setlength{\parskip}{0pt}}
\setcounter{secnumdepth}{0}
% Redefines (sub)paragraphs to behave more like sections
\ifx\paragraph\undefined\else
\let\oldparagraph\paragraph
\renewcommand{\paragraph}[1]{\oldparagraph{#1}\mbox{}}
\fi
\ifx\subparagraph\undefined\else
\let\oldsubparagraph\subparagraph
\renewcommand{\subparagraph}[1]{\oldsubparagraph{#1}\mbox{}}
\fi

%%% Use protect on footnotes to avoid problems with footnotes in titles
\let\rmarkdownfootnote\footnote%
\def\footnote{\protect\rmarkdownfootnote}

%%% Change title format to be more compact
\usepackage{titling}

% Create subtitle command for use in maketitle
\newcommand{\subtitle}[1]{
  \posttitle{
    \begin{center}\large#1\end{center}
    }
}

\setlength{\droptitle}{-2em}
  \title{Isotope Tracking}
  \pretitle{\vspace{\droptitle}\centering\huge}
  \posttitle{\par}
  \author{}
  \preauthor{}\postauthor{}
  \date{}
  \predate{}\postdate{}

\usepackage{mathtools} \usepackage{natbib}

\begin{document}
\maketitle

\subsection{Applications}\label{applications}

After a new product application at time \(t+1\), the top soil carbon
isotope signature (\(\delta ^{13}C\)) for a soil layer \(k\), is updated
by balancing mass terms. Note that for pesticide applications only the
first layer (\(k=z0\)) is considered, such that:

\[
\delta ^{13}C_{k(t+1)} = \frac{1}{M_{k,tot(t+1)}} \Big(\delta ^{13}C_{k(t)} \cdot M_{k(t)} + \delta ^{13}C_{app(t+1)} \cdot M_{app(t+1)}  \Big)
\]

\[
M_{k,tot(t+1)} =  M_{k(t)} + M_{app(t+1)}
\] where \(M_{k(t)}\) is the pesticide mass \((\mu g)\) for the layer
\(k\) present before application \(app\).

\subsection{Non-reactive transport}\label{non-reactive-transport}

For each non-fractionating mass transfer process \(\delta ^{13}C\) is
updated also by balancing mass terms for each cell:

\[
\delta ^{13}C_{k(t+1)} = \frac{1}{M_{k,tot(t+1)}} \Big(\delta ^{13}C_{k(t)} \cdot M_{k(t)} + \delta ^{13}C_{gain(t+1)} \cdot M_{gain(t+1)} -  \delta ^{13}C_{loss(t+1)} \cdot M_{loss(t+1)}  \Big)
\]

\[
M_{k,tot(t+1)} =  M_{k(t)} + M_{gain(t+1)} -  M_{loss(t+1)}
\]

Update at each cell for each layer is computed by the follwing function,
where the relevant layer and processes is selected:

\begin{Shaded}
\begin{Highlighting}[]

\KeywordTok{def} \NormalTok{update_layer_delta(model, layer, process, mass_process, mass_before_transport):}
    \ControlFlowTok{if} \NormalTok{layer }\OperatorTok{==} \DecValTok{0}\NormalTok{:}
      \NormalTok{delta_layer }\OperatorTok{=} \NormalTok{model.delta_z0}
      \NormalTok{delta_layer_above }\OperatorTok{=} \VariableTok{None}
      \NormalTok{mass_layer }\OperatorTok{=} \NormalTok{model.pestmass_z0}
    \ControlFlowTok{elif} \NormalTok{layer }\OperatorTok{==} \DecValTok{1}\NormalTok{:}
        \NormalTok{delta_layer }\OperatorTok{=} \NormalTok{model.delta_z1}
        \NormalTok{delta_layer_above }\OperatorTok{=} \NormalTok{model.delta_z0}
        \NormalTok{mass_layer }\OperatorTok{=} \NormalTok{model.pestmass_z1}
    \ControlFlowTok{elif} \NormalTok{layer }\OperatorTok{==} \DecValTok{2}\NormalTok{:}
        \NormalTok{delta_layer }\OperatorTok{=} \NormalTok{model.delta_z2}
        \NormalTok{delta_layer_above }\OperatorTok{=} \NormalTok{model.delta_z1}
        \NormalTok{mass_layer }\OperatorTok{=} \NormalTok{model.pestmass_z2}
        
    \ControlFlowTok{if} \NormalTok{process }\OperatorTok{==} \StringTok{"volat"}\NormalTok{:}
        \ControlFlowTok{pass}
    \ControlFlowTok{elif} \NormalTok{process }\OperatorTok{==} \StringTok{"runoff"}\NormalTok{:}
        \ControlFlowTok{pass}
    \ControlFlowTok{elif} \NormalTok{process }\OperatorTok{==} \StringTok{"leach"}\NormalTok{:}
        \ControlFlowTok{pass}
    \ControlFlowTok{elif} \NormalTok{process }\OperatorTok{==} \StringTok{"latflux"}\NormalTok{:}
        \ControlFlowTok{pass}
    \ControlFlowTok{else}\NormalTok{:}
        \ControlFlowTok{raise} \PreprocessorTok{NotImplementedError}
    
    \ControlFlowTok{return} \StringTok{"updated delta for delta_layer"}
                       
\end{Highlighting}
\end{Shaded}

For brevity, only two examples are shown. For volatilization, the
process process=\textbf{``volat''} is chosen and the updated
\(\delta ^{13}C\) for a soil layer \(k\) is returned as,

\begin{Shaded}
\begin{Highlighting}[]

\KeywordTok{def} \NormalTok{update_layer_delta(model, layer, process, mass_process, mass_before_transport):}
    \ControlFlowTok{if} \NormalTok{layer }\OperatorTok{==} \DecValTok{0}\NormalTok{:}
      \NormalTok{delta_layer }\OperatorTok{=} \NormalTok{model.delta_z0}
      \NormalTok{delta_layer_above }\OperatorTok{=} \DecValTok{0}
      \NormalTok{mass_layer }\OperatorTok{=} \NormalTok{model.pestmass_z0}
    \ControlFlowTok{elif} \NormalTok{layer }\OperatorTok{==} \DecValTok{1}\NormalTok{:}
        \NormalTok{delta_layer }\OperatorTok{=} \NormalTok{model.delta_z1}
        \NormalTok{delta_layer_above }\OperatorTok{=} \NormalTok{model.delta_z0}
        \NormalTok{mass_layer }\OperatorTok{=} \NormalTok{model.pestmass_z1}
    \ControlFlowTok{elif} \NormalTok{layer }\OperatorTok{==} \DecValTok{2}\NormalTok{:}
        \NormalTok{delta_layer }\OperatorTok{=} \NormalTok{model.delta_z2}
        \NormalTok{delta_layer_above }\OperatorTok{=} \NormalTok{model.delta_z1}
        \NormalTok{mass_layer }\OperatorTok{=} \NormalTok{model.pestmass_z2}
        
    \ControlFlowTok{if} \NormalTok{process }\OperatorTok{==} \StringTok{"volat"}\NormalTok{:}
        \NormalTok{mass_loss }\OperatorTok{=} \NormalTok{mass_process[}\StringTok{"mass_loss"}\NormalTok{]}
        \NormalTok{mass_gain }\OperatorTok{=} \DecValTok{0}
        \NormalTok{delta_gain }\OperatorTok{=} \DecValTok{0}
        \NormalTok{delta_loss }\OperatorTok{=} \NormalTok{delta_layer}
        
    \ControlFlowTok{elif} \NormalTok{process }\OperatorTok{==} \StringTok{"runoff"}\NormalTok{:}
        \ControlFlowTok{pass}
    \ControlFlowTok{elif} \NormalTok{process }\OperatorTok{==} \StringTok{"leach"}\NormalTok{:}
        \ControlFlowTok{pass}
    \ControlFlowTok{elif} \NormalTok{process }\OperatorTok{==} \StringTok{"latflux"}\NormalTok{:}
        \ControlFlowTok{pass}
    \ControlFlowTok{else}\NormalTok{:}
        \ControlFlowTok{raise} \PreprocessorTok{NotImplementedError}

    \ControlFlowTok{if} \NormalTok{process }\OperatorTok{==} \StringTok{"latflux"}\NormalTok{:}
        \ControlFlowTok{pass}
    \ControlFlowTok{else}\NormalTok{:}
        \NormalTok{mass_tot }\OperatorTok{=} \NormalTok{mass_before_transport }\OperatorTok{+} \NormalTok{mass_gain }\OperatorTok{-} \NormalTok{mass_loss}
        \NormalTok{delta_int }\OperatorTok{=} \NormalTok{((}\DecValTok{1}\OperatorTok{/}\NormalTok{mass_tot) }\OperatorTok{*}
                     \NormalTok{(delta_layer }\OperatorTok{*} \NormalTok{mass_before_transport }\OperatorTok{+}  \CommentTok{# initial}
                      \NormalTok{delta_gain }\OperatorTok{*} \NormalTok{mass_gain }\OperatorTok{-}  \CommentTok{# mass_in}
                      \NormalTok{delta_loss }\OperatorTok{*} \NormalTok{mass_loss))  }\CommentTok{# mass_out}
    \ControlFlowTok{return} \NormalTok{delta_int}
                       
\end{Highlighting}
\end{Shaded}

For lateral flux, update follows the same approach as that for water
mass exchange across cells. The pesticide mass at cell \(j\) after
lateral flux is given by:

\[
M_{j, tot(t+1)} = M_{j(t)} + \sum^{N(t)}_{i=1} M_{loss, i(t)} - M_{loss,j(t)} 
\]

The mass gain at cell \(j\) is given by the mass from upstream cells
\(i\) contributing to downstream cells and given by:

\[
\sum^{N(t)}_{i=1} M_{loss, i(t)} = \frac{W_{j} \sum^{N(t)}_{i=1}max[ C_{i,aq}\cdot\Big(c_{z}(SW_{i}-SW_{fc,i})\Big),~0] }{ \sum^{N(t)}_{i=1} W_{i} }
\] Loss at cell \(j\) is then given by:

\[
M_{loss,j(t)} = C_{j,aq}\cdot\Big(c_{z}(SW_{j}-SW_{fc,j})\Big)
\] Considering the isotope mass balance, while simplfying for the
relative wetness index at cell \(j\), we obtain:

\[
W_{j/i} = \frac{W_{j}}{\sum^{N(t)}_{i=1} W_{i} } 
\]

\[
\delta ^{13}C_{j(t+1)} = \frac{1}{M_{j,tot(t+1)}} 
\Big(\delta ^{13}C_{j,(t)} \cdot M_{j(t)} + 
W_{j/i} \sum^{N(t)}_{i=1}max[ \delta ^{13}C_{i} \cdot C_{i,aq}\cdot\Big(c_{z}(SW_{i}-SW_{fc,i})\Big),~0] - 
\delta ^{13}C_{j(t)} \cdot M_{loss,j(t)}  \Big)
\]

The \emph{update\_layer\_delta} functions, takes
process=\textbf{``latflux''} and implements the above equations as:

\begin{Shaded}
\begin{Highlighting}[]

\KeywordTok{def} \NormalTok{update_layer_delta(model, layer, process, mass_process, mass_before_transport):}
    \ControlFlowTok{if} \NormalTok{layer }\OperatorTok{==} \DecValTok{0}\NormalTok{:}
      \NormalTok{delta_layer }\OperatorTok{=} \NormalTok{model.delta_z0}
      \NormalTok{delta_layer_above }\OperatorTok{=} \DecValTok{0}
      \NormalTok{mass_layer }\OperatorTok{=} \NormalTok{model.pestmass_z0}
    \ControlFlowTok{elif} \NormalTok{layer }\OperatorTok{==} \DecValTok{1}\NormalTok{:}
        \NormalTok{delta_layer }\OperatorTok{=} \NormalTok{model.delta_z1}
        \NormalTok{delta_layer_above }\OperatorTok{=} \NormalTok{model.delta_z0}
        \NormalTok{mass_layer }\OperatorTok{=} \NormalTok{model.pestmass_z1}
    \ControlFlowTok{elif} \NormalTok{layer }\OperatorTok{==} \DecValTok{2}\NormalTok{:}
        \NormalTok{delta_layer }\OperatorTok{=} \NormalTok{model.delta_z2}
        \NormalTok{delta_layer_above }\OperatorTok{=} \NormalTok{model.delta_z1}
        \NormalTok{mass_layer }\OperatorTok{=} \NormalTok{model.pestmass_z2}
        
    \ControlFlowTok{if} \NormalTok{process }\OperatorTok{==} \StringTok{"volat"}\NormalTok{:}
        \ControlFlowTok{pass}
    \ControlFlowTok{elif} \NormalTok{process }\OperatorTok{==} \StringTok{"runoff"}\NormalTok{:}
        \ControlFlowTok{pass}
    \ControlFlowTok{elif} \NormalTok{process }\OperatorTok{==} \StringTok{"leach"}\NormalTok{:}
        \ControlFlowTok{pass}
    \ControlFlowTok{elif} \NormalTok{process }\OperatorTok{==} \StringTok{"latflux"}\NormalTok{:}
        \NormalTok{mass_latflux }\OperatorTok{=} \NormalTok{mass_process[}\StringTok{"net_mass_latflux"}\NormalTok{]  }\CommentTok{# mg}
        \NormalTok{mass_loss }\OperatorTok{=} \NormalTok{mass_process[}\StringTok{"cell_mass_loss_downstream"}\NormalTok{]}
        \NormalTok{mass_gain }\OperatorTok{=} \NormalTok{mass_process[}\StringTok{"upstream_mass_inflow"}\NormalTok{]}
        \NormalTok{mass_tot }\OperatorTok{=} \NormalTok{mass_before_transport }\OperatorTok{+} \NormalTok{mass_gain }\OperatorTok{-} \NormalTok{mass_loss}
        \CommentTok{# Proof of first fraction, i.e., f1 > 1}
        \NormalTok{f1 }\OperatorTok{=} \NormalTok{mass_before_transport }\OperatorTok{/} \NormalTok{mass_tot}
        \CommentTok{# Proof of 2nd (inflow) fraction, f2 < 1}
        \CommentTok{# f2 = accuflux(model.ldd, mass_gain)/accuflux(model.ldd, mass_tot)}
        \NormalTok{f3 }\OperatorTok{=} \NormalTok{mass_loss }\OperatorTok{/} \NormalTok{mass_tot  }\CommentTok{# Proof of third (leaving) mass fraction, f3 < 1}
    \ControlFlowTok{else}\NormalTok{:}
        \ControlFlowTok{raise} \PreprocessorTok{NotImplementedError}

    \ControlFlowTok{if} \NormalTok{process }\OperatorTok{==} \StringTok{"latflux"}\NormalTok{:}
        \NormalTok{delta2_f2 }\OperatorTok{=} \NormalTok{accuflux(model.ldd_subs, mass_gain}\OperatorTok{*}\NormalTok{delta_layer)}\OperatorTok{/}\NormalTok{accuflux(model.ldd_subs, mass_tot)}
        \NormalTok{delta_int }\OperatorTok{=} \NormalTok{(delta_layer }\OperatorTok{*} \NormalTok{f1) }\OperatorTok{+} \NormalTok{delta2_f2 }\OperatorTok{-} \NormalTok{(delta_layer }\OperatorTok{*} \NormalTok{f3)}
    \ControlFlowTok{else}\NormalTok{:}
        \ControlFlowTok{pass}
        
    \ControlFlowTok{return} \NormalTok{delta_int}
                       
\end{Highlighting}
\end{Shaded}

Full function:

\begin{Shaded}
\begin{Highlighting}[]

\KeywordTok{def} \NormalTok{getLatMassDeltaFlux(model, layer, theta_sat, theta_fcap,}
                        \NormalTok{mass_before_transport, sorption_model}\OperatorTok{=}\StringTok{'linear'}\NormalTok{, gas}\OperatorTok{=}\VariableTok{True}\NormalTok{):}
    \ControlFlowTok{if} \NormalTok{layer }\OperatorTok{==} \DecValTok{0}\NormalTok{:}
        \NormalTok{depth }\OperatorTok{=} \NormalTok{model.z0}
        \NormalTok{delta_layer }\OperatorTok{=} \NormalTok{model.delta_z0}
        \NormalTok{theta_layer }\OperatorTok{=} \NormalTok{model.theta_z0}
        \NormalTok{mass_layer }\OperatorTok{=} \NormalTok{model.pestmass_z0}
        \NormalTok{c }\OperatorTok{=} \NormalTok{model.c1}
    \ControlFlowTok{elif} \NormalTok{layer }\OperatorTok{==} \DecValTok{1}\NormalTok{:}
        \NormalTok{depth }\OperatorTok{=} \NormalTok{model.z1}
        \NormalTok{delta_layer }\OperatorTok{=} \NormalTok{model.delta_z1}
        \NormalTok{theta_layer }\OperatorTok{=} \NormalTok{model.theta_z1}
        \NormalTok{mass_layer }\OperatorTok{=} \NormalTok{model.pestmass_z1}
        \NormalTok{c }\OperatorTok{=} \NormalTok{model.c1}
    \ControlFlowTok{elif} \NormalTok{layer }\OperatorTok{==} \DecValTok{2}\NormalTok{:}
        \NormalTok{store }\OperatorTok{=} \VariableTok{False}
        \NormalTok{depth }\OperatorTok{=} \NormalTok{model.z2}
        \NormalTok{delta_layer }\OperatorTok{=} \NormalTok{model.delta_z2}
        \NormalTok{theta_layer }\OperatorTok{=} \NormalTok{model.theta_z2}
        \NormalTok{mass_layer }\OperatorTok{=} \NormalTok{model.pestmass_z2}
        \NormalTok{c }\OperatorTok{=} \NormalTok{model.c2}

    \ControlFlowTok{if} \NormalTok{sorption_model }\OperatorTok{==} \StringTok{"linear"}\NormalTok{:}
        \CommentTok{# Retardation factor}
        \NormalTok{retard_layer }\OperatorTok{=} \DecValTok{1} \OperatorTok{+} \NormalTok{(model.p_b }\OperatorTok{*} \NormalTok{model.k_d) }\OperatorTok{/} \NormalTok{theta_layer}
    \ControlFlowTok{else}\NormalTok{:}
        \NormalTok{retard_layer }\OperatorTok{=} \DecValTok{1}

    \ControlFlowTok{if} \NormalTok{gas:}
        \CommentTok{# Leistra et al., 2001}
        \NormalTok{theta_gas }\OperatorTok{=} \NormalTok{theta_sat }\OperatorTok{-} \NormalTok{theta_layer}
        \NormalTok{conc_layer_aq }\OperatorTok{=} \NormalTok{mass_layer }\OperatorTok{/} \NormalTok{((cellarea() }\OperatorTok{*} \NormalTok{depth) }\OperatorTok{*}
                                      \NormalTok{(theta_gas }\OperatorTok{*} \NormalTok{model.k_h }\OperatorTok{+} \NormalTok{theta_layer }\OperatorTok{*} \NormalTok{retard_layer))  }\CommentTok{# mg/L}
    \ControlFlowTok{else}\NormalTok{:}
        \CommentTok{# Whelan, 1987 # No gas phase considered}
        \NormalTok{conc_layer_aq }\OperatorTok{=} \NormalTok{(mass_layer }\OperatorTok{/} \NormalTok{cellarea()) }\OperatorTok{/} \NormalTok{(theta_layer }\OperatorTok{*} \NormalTok{retard_layer }\OperatorTok{*} \NormalTok{depth)  }\CommentTok{# mg/L}

    \CommentTok{# W(j/i)}
    \NormalTok{rel_wetness }\OperatorTok{=} \NormalTok{model.wetness }\OperatorTok{/} \NormalTok{accuflux(model.ldd_subs, model.wetness)}

    \CommentTok{# Cell mass loss/gain (to update only mass)}
    \NormalTok{mass_loss }\OperatorTok{=} \BuiltInTok{max}\NormalTok{(conc_layer_aq }\OperatorTok{*} \NormalTok{(c }\OperatorTok{*} \NormalTok{(depth }\OperatorTok{*} \NormalTok{theta_layer }\OperatorTok{-} \NormalTok{depth }\OperatorTok{*} \NormalTok{theta_fcap)), scalar(}\DecValTok{0}\NormalTok{))}
    \NormalTok{mass_gain }\OperatorTok{=} \NormalTok{rel_wetness }\OperatorTok{*} \NormalTok{accuflux(model.ldd_subs, mass_loss)}
    \NormalTok{net_mass_latflux }\OperatorTok{=} \NormalTok{mass_gain }\OperatorTok{-} \NormalTok{mass_loss}

    \CommentTok{# massDelta  loss/gain (to update only delta)}
    \NormalTok{massdC_loss }\OperatorTok{=} \BuiltInTok{max}\NormalTok{(delta_layer }\OperatorTok{*} \NormalTok{conc_layer_aq }\OperatorTok{*} \NormalTok{(c }\OperatorTok{*} \NormalTok{(depth }\OperatorTok{*} \NormalTok{theta_layer }\OperatorTok{-} \NormalTok{depth }\OperatorTok{*} \NormalTok{theta_fcap)), scalar(}\DecValTok{0}\NormalTok{))}
    \NormalTok{massdC_gain }\OperatorTok{=} \NormalTok{rel_wetness }\OperatorTok{*} \NormalTok{accuflux(model.ldd_subs, massdC_loss)}

    \CommentTok{# Isotope mass balance update three terms:}
    \NormalTok{mass_tot }\OperatorTok{=} \NormalTok{mass_before_transport }\OperatorTok{+} \NormalTok{mass_gain }\OperatorTok{-} \NormalTok{mass_loss}
    \CommentTok{# 1st fraction term (f1), must: f1 > 1, with net gain}
    \NormalTok{f1 }\OperatorTok{=} \NormalTok{(mass_before_transport }\OperatorTok{/} \NormalTok{mass_tot) }\OperatorTok{*} \NormalTok{delta_layer}
    \NormalTok{f2 }\OperatorTok{=} \NormalTok{massdC_gain }\OperatorTok{/} \NormalTok{mass_tot}
    \NormalTok{f3 }\OperatorTok{=} \NormalTok{massdC_loss }\OperatorTok{/} \NormalTok{mass_tot}
    \NormalTok{delta_layer }\OperatorTok{=} \NormalTok{f1 }\OperatorTok{+} \NormalTok{f2 }\OperatorTok{-} \NormalTok{f3}

    \ControlFlowTok{return} \NormalTok{\{}
        \StringTok{'mass_loss'}\NormalTok{: mass_loss,  }\CommentTok{# mg}
        \CommentTok{'mass_gain'}\NormalTok{: mass_gain,  }\CommentTok{# mg}
        \CommentTok{'net_mass_latflux'}\NormalTok{: net_mass_latflux,  }\CommentTok{# mg}
        \CommentTok{'massdC_loss'}\NormalTok{: massdC_loss,  }\CommentTok{# mg*dC}
        \CommentTok{'massdC_gain'}\NormalTok{: massdC_gain,  }\CommentTok{# mg*dC}
        \CommentTok{'f1'}\NormalTok{: f1,  }\CommentTok{# could be > 1  (if net gain)}
        \CommentTok{'f2'}\NormalTok{: f2,}
        \CommentTok{'f3'}\NormalTok{: f3,}
        \CommentTok{'delta_layer'}\NormalTok{: delta_layer}
    \NormalTok{\}}
                       
\end{Highlighting}
\end{Shaded}

\bibliography{library.bib}


\end{document}
