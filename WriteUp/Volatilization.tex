\documentclass[]{article}
\usepackage{lmodern}
\usepackage{amssymb,amsmath}
\usepackage{ifxetex,ifluatex}
\usepackage{fixltx2e} % provides \textsubscript
\ifnum 0\ifxetex 1\fi\ifluatex 1\fi=0 % if pdftex
  \usepackage[T1]{fontenc}
  \usepackage[utf8]{inputenc}
\else % if luatex or xelatex
  \ifxetex
    \usepackage{mathspec}
  \else
    \usepackage{fontspec}
  \fi
  \defaultfontfeatures{Ligatures=TeX,Scale=MatchLowercase}
\fi
% use upquote if available, for straight quotes in verbatim environments
\IfFileExists{upquote.sty}{\usepackage{upquote}}{}
% use microtype if available
\IfFileExists{microtype.sty}{%
\usepackage{microtype}
\UseMicrotypeSet[protrusion]{basicmath} % disable protrusion for tt fonts
}{}
\usepackage[margin=1in]{geometry}
\usepackage{hyperref}
\hypersetup{unicode=true,
            pdftitle={Volatilization Alternative explanation},
            pdfborder={0 0 0},
            breaklinks=true}
\urlstyle{same}  % don't use monospace font for urls
\usepackage{natbib}
\bibliographystyle{plainnat}
\usepackage{graphicx,grffile}
\makeatletter
\def\maxwidth{\ifdim\Gin@nat@width>\linewidth\linewidth\else\Gin@nat@width\fi}
\def\maxheight{\ifdim\Gin@nat@height>\textheight\textheight\else\Gin@nat@height\fi}
\makeatother
% Scale images if necessary, so that they will not overflow the page
% margins by default, and it is still possible to overwrite the defaults
% using explicit options in \includegraphics[width, height, ...]{}
\setkeys{Gin}{width=\maxwidth,height=\maxheight,keepaspectratio}
\IfFileExists{parskip.sty}{%
\usepackage{parskip}
}{% else
\setlength{\parindent}{0pt}
\setlength{\parskip}{6pt plus 2pt minus 1pt}
}
\setlength{\emergencystretch}{3em}  % prevent overfull lines
\providecommand{\tightlist}{%
  \setlength{\itemsep}{0pt}\setlength{\parskip}{0pt}}
\setcounter{secnumdepth}{0}
% Redefines (sub)paragraphs to behave more like sections
\ifx\paragraph\undefined\else
\let\oldparagraph\paragraph
\renewcommand{\paragraph}[1]{\oldparagraph{#1}\mbox{}}
\fi
\ifx\subparagraph\undefined\else
\let\oldsubparagraph\subparagraph
\renewcommand{\subparagraph}[1]{\oldsubparagraph{#1}\mbox{}}
\fi

%%% Use protect on footnotes to avoid problems with footnotes in titles
\let\rmarkdownfootnote\footnote%
\def\footnote{\protect\rmarkdownfootnote}

%%% Change title format to be more compact
\usepackage{titling}

% Create subtitle command for use in maketitle
\newcommand{\subtitle}[1]{
  \posttitle{
    \begin{center}\large#1\end{center}
    }
}

\setlength{\droptitle}{-2em}
  \title{Volatilization Alternative explanation}
  \pretitle{\vspace{\droptitle}\centering\huge}
  \posttitle{\par}
  \author{}
  \preauthor{}\postauthor{}
  \date{}
  \predate{}\postdate{}

\usepackage{mathtools} \usepackage{natbib}

\begin{document}
\maketitle

Transport processes, including volatilization, leaching, lateral mass
flux are considered sequentially before degradation and direclty after
application. Sorption is considered within each process.

\hypertarget{volatilization}{%
\subsection{Volatilization}\label{volatilization}}

Volatilization is assumed only to take only place on the application
day, and follows \cite{Leistra2001} by considering mass flux across two
boundary layers. Mass flux across an air boundary \(J_{v,a}\)
\((\mu g/ m^2d)\) and mass flux across the top soil layer \(J_{v,s}\)
\((\mu g/ m^2d)\) and given by:

\[.
J_{v,air} = \frac{-(C_{g,0(t)}-C_{air})}{r_{air(t)}}
\]

\[.
J_{v,soil} = \frac{-(C_{g,z0}-C_{g,0})}{r_{soil(t)}}
\] where, \(r_a\) \((d/m)\) is the air boundary resistance and \(r_s\)
\((d/m)\) the top soil layer resistance before the pesticide reaches the
soil surface from the center of the cell layer \((0.5 \cdot D_{z0})\).
Concentration gradients \((\mu g/m^3)\) are considered based on the
concentration at the soil surface \(C_{g,0}\), the air boundary layer
\(C_{air}\) and at the center of top soil layer.

Resistance across the boundary air and soil layers are given
respectively by:

\[.
r_{air(t)} = \frac{D_{z,air}}{D_{a(t)}} 
\]

\[.
r_{soil(t)} = \frac{0.5 \cdot D_{z0}}{D_{diff,g}} 
\]

where \(D_{z,air}\) \((m)\) is the boundary air layer assumed to be
equal to the the first soil layer \((k=z0)\) thickness and the pesticide
diffusion coefficient \(D_a\) \((m^2/d)\), which is dependent on the
daily air temperature \(T\) (in \(K\)) as:

\[.
D_{a(t)} = \Big( \frac{T}{T_r}\Big)^{1.75}D_{a,r}
\] and where \(D_{a,r}\) \((m^2/d)\), is the diffussion coefficient in
air at reference temperature \(T_r\). Two options are available to
compute the relative diffussion coefficient \(D_{diff,g}\) \((m^2/d)\),
which is adjusted to the available pore air space
\(\theta_{z0,air(t)} = \theta_{z0,sat}-\theta_{z0(t)}\). The first
option is given by \citep{Millington1960, Jin1996}:

\[.
D_{diff,g(t)} = \frac{D_{a(t)}\cdot (\theta_{z0,air(t)})^2}{(\theta_{z0,sat})^{2/3}}
\]

The second option is given by \citep{Currie1960, Bakker1987}:

\[.
D_{diff,g(t)} = 2.5 \cdot D_{a(t)}\cdot (\theta_{z0,air(t)})^3
\]

During each process, sorption extent is computed (as shown above) to
obtain the concentration in aqueous \(C_{z0,aq}\) \((\mu g/L)\) and gas
\(C_{z0,g}\) \((\mu g/L)\) solutions. It is assumed that the diffusion
flux from the top soil layer is equal to the diffusion flux across the
bounadry air layer and that concentration in air \(C_{air}=0\),
yielding:

\begin{equation}
C_{g,0} = \frac{r_{air(t)}}{r_{air(t)} + r_{soil(t)}} \cdot C_{g,z0}
\label{eq:Cgas0} 
\end{equation}

Substituting eq. \ref{eq:Cgas0} into \ref{eq:Jvair} \ldots{}

The volatilized flux out of the top layer is given by:

\ldots{}this has been documented in BEACH Methodoogy!!

\bibliography{library.bib}


\end{document}
